%++++++++++++++++++++++++++++++++++++++++
\documentclass[article, 12pt]{article}
\usepackage{float}
\usepackage{setspace}
\usepackage{tabu} % extra features for tabular environment
\usepackage{amsmath}  % improve math presentation
\usepackage{graphicx} % takes care of graphic including machinery
\usepackage[margin=1in]{geometry} % decreases margins
\usepackage{cite} % takes care of citations
\usepackage[final]{hyperref} % adds hyper links inside the generated pdf file
\usepackage{tikz}
\usepackage{caption} 
\usepackage{fancyhdr}
\usepackage{amssymb} % symbols like /therefore
\usepackage{amsthm} % proofs
\usepackage{enumerate} % lettered lists
\usepackage{mathtools} % macros
\usetikzlibrary{scopes}
% \usepackage{xcolor} \pagecolor[rgb]{0.12549019607,0.1294117647,0.13725490196} \color[rgb]{0.82352941176,0.76862745098,0.62745098039} % dark theme
\theoremstyle{definition}
\newtheorem{example}{Example}[subsubsection]
\newtheorem*{remark}{Remark}
\newtheorem{theorem}{Theorem}[subsubsection]
\newtheorem{definition}{Definition}[subsubsection]
\newtheorem{corollary}{Corollary}[subsubsection]
\hypersetup{
	colorlinks=false,      % false: boxed links; true: colored links
	linkcolor=blue,        % color of internal links
	citecolor=blue,        % color of links to bibliography
	filecolor=magenta,     % color of file links
	urlcolor=blue         
}
\usepackage{physics}
\usepackage{siunitx}
\usepackage{tikz,pgfplots}
\usepackage[outline]{contour} % glow around text
\usetikzlibrary{calc}
\usetikzlibrary{angles,quotes} % for pic
\usetikzlibrary{arrows.meta}
\tikzset{>=latex} % for LaTeX arrow head
\contourlength{1.2pt}

\colorlet{xcol}{blue!70!black}
\colorlet{vcol}{green!60!black}
\colorlet{myred}{red!70!black}
\colorlet{myblue}{blue!70!black}
\colorlet{mygreen}{green!70!black}
\colorlet{mydarkred}{myred!70!black}
\colorlet{mydarkblue}{myblue!60!black}
\colorlet{mydarkgreen}{mygreen!60!black}
\colorlet{acol}{red!50!blue!80!black!80}
\tikzstyle{CM}=[red!40!black,fill=red!80!black!80]
\tikzstyle{xline}=[xcol,thick,smooth]
\tikzstyle{mass}=[line width=0.6,red!30!black,fill=red!40!black!10,rounded corners=1,
                  top color=red!40!black!20,bottom color=red!40!black!10,shading angle=20]
\tikzstyle{faded mass}=[dashed,line width=0.1,red!30!black!40,fill=red!40!black!10,rounded corners=1,
                        top color=red!40!black!10,bottom color=red!40!black!10,shading angle=20]
\tikzstyle{rope}=[brown!70!black,very thick,line cap=round]
\def\rope#1{ \draw[black,line width=1.4] #1; \draw[rope,line width=1.1] #1; }
\tikzstyle{force}=[->,myred,very thick,line cap=round]
\tikzstyle{velocity}=[->,vcol,very thick,line cap=round]
\tikzstyle{Fproj}=[force,myred!40]
\tikzstyle{myarr}=[-{Latex[length=3,width=2]},thin]
\def\tick#1#2{\draw[thick] (#1)++(#2:0.12) --++ (#2-180:0.24)}
\DeclareMathOperator{\sn}{sn}
\DeclareMathOperator{\cn}{cn}
\DeclareMathOperator{\dn}{dn}
\def\N{80} % number of samples in plots


\usepackage{titling}
\renewcommand\maketitlehooka{\null\mbox{}\vfill}
\renewcommand\maketitlehookd{\vfill\null}
\usepackage{siunitx} % units
\usepackage{verbatim} 
\newcommand{\courseNumber}{MATH 1700}
\newcommand{\courseName}{Ideas in Mathematics}
\newcommand{\professor}{Professor Rimmer}
\newcommand{\psetName}{Portfolio Question 12}
\newcommand{\dueDate}{Due: February 10, 2023}
\newcommand{\name}{Denny Cao}
\pagestyle{fancy}
\fancyhf{}% clears all header and footer fields
\fancyfoot[C]{--~\thepage~--}
\renewcommand*{\headrulewidth}{0.4pt}
\renewcommand*{\footrulewidth}{0pt}
\lhead{\name}
\chead{\courseNumber: \courseName}
\rhead{\professor}


\fancypagestyle{plain}{%
  \fancyhf{}% clears all header and footer fields
  \fancyfoot[C]{--~\thepage~--}%
  \renewcommand*{\headrulewidth}{0pt}%
  \renewcommand*{\footrulewidth}{0pt}%
}

% Shortcuts
\DeclarePairedDelimiter\ceil{\lceil}{\rceil} % ceil function
\DeclarePairedDelimiter\floor{\lfloor}{\rfloor} % floor function

\DeclarePairedDelimiter\paren{(}{)} % parenthesis

\newcommand{\df}{\displaystyle\frac} % displaystyle fraction
\newcommand{\qeq}{\overset{?}{=}} % questionable equality

\newcommand{\Mod}[1]{\;\mathrm{mod}\; #1} % modulo operator

% Sets
\DeclarePairedDelimiter\set{\{}{\}}
\newcommand{\unite}{\cup}
\newcommand{\inter}{\cap}

\newcommand{\reals}{\mathbb{R}} % real numbers: textbook is Z^+ and 0
\newcommand{\ints}{\mathbb{Z}}
\newcommand{\nats}{\mathbb{N}}
\newcommand{\rats}{\mathbb{Q}}

\newcommand{\degree}{^\circ}

% Counting
\newcommand\perm[2][^n]{\prescript{#1\mkern-2.5mu}{}P_{#2}}
\newcommand\comb[2][^n]{\prescript{#1\mkern-0.5mu}{}C_{#2}}

\setlength\parindent{0pt}

% Card Symbols
\newcommand{\clubs}{\clubsuit}
\newcommand{\diamonds}{\vardiamondsuit}
\newcommand{\hearts}{\varheartsuit}
\newcommand{\spades}{\spadesuit}

%++++++++++++++++++++++++++++++++++++++++
% title stuff

\makeatletter
\renewcommand{\maketitle}{\bgroup\setlength{\parindent}{0pt}
    \begin{flushleft}
        \textbf{\@title} \\ \vskip0.2cm
        \begingroup
            \fontsize{14pt}{12pt}\selectfont
            \courseNumber: \courseName 
            \vskip0.3cm 
            \professor
        \endgroup \vskip0.3cm
        \@date \hfill\rlap{}\bf{\name} \\ \vskip0.1cm
        \hrulefill
    \end{flushleft}\egroup 
}
\makeatother

\title{\LARGE\bf{\psetName}}
\author{\name}
\date{\dueDate}

\author{\name}
\date{\dueDate}

\begin{document}
    \maketitle
    \thispagestyle{empty}
    \section*{Question 12}
    \doublespacing
    \textbf{Explain why \textit{each} of the numbers in the list}
    \[ (1 \cdot 2 \cdot 3 \cdot 4) + 2, (1 \cdot 2 \cdot 3 \cdot 4) + 3, (1 \cdot 2 \cdot 3 \cdot 4) + 4\]
    \textbf{is composite. (You can show that they are composite by performing all of the relevant additions and multiplications, but it might help to think about how he way the numbers in the list were constructed guarantees that they are composite.)}
    \\[12pt]
    \textbf{Next show that for any natural number $n$, we can find a natural number $m$ so that the numbers $m+1,m+2,\dots,m+n$ are all composite. (In other words, for every natural number $n$, we can find $n$ \textit{consecutive} natural numbers which are all composite.) Be very careful with your choice of $m$. You might need to try out some examples to make sure your construction really works the way you want it to, even if you don't end up including those extra examples in your write-up.}
    \pagebreak
    \section*{First Submission}
    \subsection*{Part 1}
    \begin{proof}
        Let $a = (1 \cdot 2 \cdot 3 \cdot 4)$. For each number in the list, the number added to $a$ is a factor of $a$. Because of this, we can factor out the the number added to $a$ from both. We can rewrite each number in the list as follows:
        \[ 2(1 \cdot 3 \cdot 4 + 1), 3(1 \cdot 2 \cdot 4 + 1), 4(1 \cdot 2 \cdot 3 + 1) \]
        As each of these numbers are multiples of $2,3,$ and $4$, respectively, they are all composite.
    \end{proof}
    \subsection*{Part 2}
    When  
    \[ m = \prod_{i=1}^{n} i \]
    the numbers $m+1,m+2,\dots,m+n$ are all composite.

    \begin{proof}
        As $m$ is defined as the product of the first $n$ natural numbers, each number from 1 to $n$ is a factor of $n$. Thus, $\forall x \in \nats (x \leq n \to x \mid m)$. Because of this, we can factor out each number from 1 to $n$ from $m$. We can rewrite the list of numbers from $m+1 \dots m+n$ as follows:
        \[ 1\paren*{\frac{m}{1}+1}, 2\paren*{\frac{m}{2} + 1}, 3\paren*{\frac{m}{3} + 1}, \dots, n\paren*{\frac{m}{n} + 1} \]
        As $\forall x \in \nats (x \leq n \to x \mid m)$, $\displaystyle\frac{m}{x} \in \nats$. As each number in the list is a multiple of $x$, they are all composite.
    \end{proof}
    \pagebreak
    \section*{Peer Review}
    \begin{figure}[H]
        \centering
        \begin{tabular}{|p{6cm}|p{4cm}|p{6cm}|} 
            \hline \textbf{Suggestions} & \textbf{Communications} & \textbf{Strengths}\\
            Part 1 seems good, but for part 2 you might have to double-check whether the statement that $m + n$ is composite for all n is true. Maybe try plugging in numbers to check? An example might also help solidify the explanation a little bit.
            & \textbf{H} Show All Steps & Everything is explained well either through words or through math, nice explanations!\\
            \hline
            Perhaps explaining why a number is composite if it is a multiple of a number that is not 1? It sounds very elementary but I think it might be needed. & \textbf{H} Explain Why, Not Just What & Nicely explained through mathematical work, and it's easy to understand. \\
            \hline
            No suggestions & \textbf{C} Avoid Ambiguous Pronouns & Not much to say, good wording. \\
            \hline
            Perhaps define composite? Otherwise, it all seems good. & \textbf{C} Use Terminology Correctly & Cool use of the product operator and set theory! \\
            \hline
            Quite short explanations, however, they convey everything in a clear and concise manner. Maybe provide a little more explanation on how each solution is correct? & \textbf{H} Include Appropriate Level of Details & Nice and straightforward explanations, cool use of the product operator! \\
            \hline
            I don't think you need any diagrams for this problem, and your explanation doesn't really need a diagram. & \textbf{H} Create Useful Diagrams & N/A \\
            \hline
            No Suggestions. & \textbf{H} Use Clear and Direct Sentences & Sentences are all clear. \\
            \hline
            No suggestions. & \textbf{C} Check Problem Setup & Setup seems good \\
            \hline
            No suggestions. & \textbf{C} Check Your Calculations & All good! \\
            \hline
            Everything seems good, except for the statement that 
$1(\frac{m}{1} + 1)$ must be composite because having a factor of 1 does not prove that the number is composite. & \textbf{C} Verify Answer is Reasonable & A well-reasoned answer, just the problem with the $1(\frac{m}{1} + 1)$ statement. \\
            \hline 
            \end{tabular} 
    \end{figure}
    \pagebreak
    \section*{Final Submission}
    \subsection*{Part 1}
    \begin{proof}
        Let $a = (1 \cdot 2 \cdot 3 \cdot 4)$. For each number in the list, the number added to $a$ is a factor of $a$. Because of this, we can factor out the the number added to $a$ from both. We can rewrite each number in the list as follows:
        \[ 2(1 \cdot 3 \cdot 4 + 1), 3(1 \cdot 2 \cdot 4 + 1), 4(1 \cdot 2 \cdot 3 + 1) \]
        The definition of a composite number is a number that has factors other than 1 and itself. As each of these numbers are multiples of 2,3, and 4 respectively, they are all composite.
    \end{proof}
    \subsection*{Part 2}
    When 
    \[ m = (n+2)! + 2 \]

    \begin{proof}
        $m$ is the product of the first $n+2$ natural numbers plus 2, or $m = (n+2)! + 2$. Note that, as $(n+2)!$ is the product of the first $n+2$ natural numbers, each number from 1 to $n+2$ is a factor of $(n+2)!$. As the least value of $n$ is 1, we can always factor out 2 from $m$.
        \\[12pt]
        Consider the list of numbers $m, m+1, m+2, \dots, m+n$. We can rewrite this list by factoring as follows:
        \begin{align*}
            m &= (n+2)! + 2 = 2\paren*{\frac{(n+2)!}{2} + 1} \\
            m+1 &= (n+2)! + (2+1) = 3\paren*{\frac{(n+2)!}{3} + 1} \\
            m+2 &= (n+2)! + (2+2) = 4\paren*{\frac{(n+2)!}{4} + 1} \\
            &\vdotswithin{=} \\
            m+n &= (n+2)! + (n+2) = (n+2)\paren*{\frac{(n+2)!}{n+2} + 1}
        \end{align*}
        Note that, when the multiple is $n$, it is multiplied by a number other than 1, as $(n+2)!/n + 1$ will result in a number greater than 1. 
        \\[12pt]
        The definition of a composite natural number is a number that has a factor other than 1 and itself. As each number in the list is a multiple of $2,3,\dots, n+2$, they are all composite. 
    \end{proof}
\end{document}
