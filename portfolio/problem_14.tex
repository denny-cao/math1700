%++++++++++++++++++++++++++++++++++++++++
\documentclass[article, 12pt]{article}
\usepackage{float}
\usepackage{setspace}
\usepackage{tabu} % extra features for tabular environment
\usepackage{amsmath}  % improve math presentation
\usepackage{graphicx} % takes care of graphic including machinery
\usepackage[margin=1in]{geometry} % decreases margins
\usepackage{cite} % takes care of citations
\usepackage[final]{hyperref} % adds hyper links inside the generated pdf file
\usepackage{tikz}
\usepackage{caption} 
\usepackage{fancyhdr}
\usepackage{amssymb} % symbols like /therefore
\usepackage{amsthm} % proofs
\usepackage{enumerate} % lettered lists
\usepackage{mathtools} % macros
\usepackage[ all]{xy} % for diagrams
\usepackage{tkz-graph}
\usetikzlibrary{knots}
\usetikzlibrary{scopes}
% \usepackage{xcolor} \pagecolor[rgb]{0.12549019607,0.1294117647,0.13725490196} \color[rgb]{0.82352941176,0.76862745098,0.62745098039} % dark theme
\theoremstyle{definition}
\newtheorem{example}{Example}[subsubsection]
\newtheorem*{remark}{Remark}
\newtheorem{theorem}{Theorem}
\newtheorem{definition}{Definition}[subsubsection]
\newtheorem{corollary}{Corollary}[subsubsection]
\hypersetup{
	colorlinks=true,      % false: boxed links; true: colored links
	linkcolor=blue,        % color of internal links
	citecolor=blue,        % color of links to bibliography
	filecolor=magenta,     % color of file links
	urlcolor=blue         
}
\usepackage{physics}
\usepackage{siunitx}
\usepackage{tikz,pgfplots}
\usepackage[outline]{contour} % glow around text
\usetikzlibrary{calc}
\usetikzlibrary{angles,quotes} % for pic
\usetikzlibrary{arrows.meta}
\tikzset{>=latex} % for LaTeX arrow head
\contourlength{1.2pt}

\colorlet{xcol}{blue!70!black}
\colorlet{vcol}{green!60!black}
\colorlet{myred}{red!70!black}
\colorlet{myblue}{blue!70!black}
\colorlet{mygreen}{green!70!black}
\colorlet{mydarkred}{myred!70!black}
\colorlet{mydarkblue}{myblue!60!black}
\colorlet{mydarkgreen}{mygreen!60!black}
\colorlet{acol}{red!50!blue!80!black!80}
\tikzstyle{CM}=[red!40!black,fill=red!80!black!80]
\tikzstyle{xline}=[xcol,thick,smooth]
\tikzstyle{mass}=[line width=0.6,red!30!black,fill=red!40!black!10,rounded corners=1,
                  top color=red!40!black!20,bottom color=red!40!black!10,shading angle=20]
\tikzstyle{faded mass}=[dashed,line width=0.1,red!30!black!40,fill=red!40!black!10,rounded corners=1,
                        top color=red!40!black!10,bottom color=red!40!black!10,shading angle=20]
\tikzstyle{rope}=[brown!70!black,very thick,line cap=round]
\def\rope#1{ \draw[black,line width=1.4] #1; \draw[rope,line width=1.1] #1; }
\tikzstyle{force}=[->,myred,very thick,line cap=round]
\tikzstyle{velocity}=[->,vcol,very thick,line cap=round]
\tikzstyle{Fproj}=[force,myred!40]
\tikzstyle{myarr}=[-{Latex[length=3,width=2]},thin]
\def\tick#1#2{\draw[thick] (#1)++(#2:0.12) --++ (#2-180:0.24)}
\DeclareMathOperator{\sn}{sn}
\DeclareMathOperator{\cn}{cn}
\DeclareMathOperator{\dn}{dn}
\def\N{80} % number of samples in plots


\usepackage{titling}
\renewcommand\maketitlehooka{\null\mbox{}\vfill}
\renewcommand\maketitlehookd{\vfill\null}
\usepackage{siunitx} % units
\usepackage{verbatim} 
\newcommand{\courseNumber}{MATH 1700}
\newcommand{\courseName}{Ideas in Mathematics}
\newcommand{\professor}{Professor Rimmer}
\newcommand{\psetName}{Portfolio Resubmission 2}
\newcommand{\dueDate}{Due: March 31, 2023}
\newcommand{\name}{Denny Cao}
\pagestyle{fancy}
\fancyhf{}% clears all header and footer fields
\fancyfoot[C]{--~\thepage~--}
\renewcommand*{\headrulewidth}{0.4pt}
\renewcommand*{\footrulewidth}{0pt}
\lhead{\name}
\chead{\courseNumber: \courseName}
\rhead{\professor}

% new theorem for questions and answers

\newtheorem{question}{Question}

\newtheorem{answer}{Answer}

\fancypagestyle{plain}{%
  \fancyhf{}% clears all header and footer fields
  \fancyfoot[C]{--~\thepage~--}%
  \renewcommand*{\headrulewidth}{0pt}%
  \renewcommand*{\footrulewidth}{0pt}%
}

% Shortcuts
\DeclarePairedDelimiter\ceil{\lceil}{\rceil} % ceil function
\DeclarePairedDelimiter\floor{\lfloor}{\rfloor} % floor function

\DeclarePairedDelimiter\paren{(}{)} % parenthesis

\newcommand{\df}{\displaystyle\frac} % displaystyle fraction
\newcommand{\qeq}{\overset{?}{=}} % questionable equality

\newcommand{\Mod}[1]{\;\mathrm{mod}\; #1} % modulo operator

\newcommand{\comp}{\circ} % composition

% Sets
\DeclarePairedDelimiter\set{\{}{\}}
\newcommand{\unite}{\cup}
\newcommand{\inter}{\cap}

\newcommand{\reals}{\mathbb{R}} % real numbers: textbook is Z^+ and 0
\newcommand{\ints}{\mathbb{Z}}
\newcommand{\nats}{\mathbb{N}}
\newcommand{\rats}{\mathbb{Q}}

\newcommand{\degree}{^\circ}

% Counting
\newcommand\perm[2][^n]{\prescript{#1\mkern-2.5mu}{}P_{#2}}
\newcommand\comb[2][^n]{\prescript{#1\mkern-0.5mu}{}C_{#2}}

% Relations
\newcommand{\rel}{\mathcal{R}} % relation

\setlength\parindent{0pt}

% Directed Graphs
\usetikzlibrary{arrows}
\tikzset{vertex/.style = {shape=circle,draw, minimum size=1.5em,
inner sep=0pt, outer sep=0pt}}
\tikzset{edge/.style = {->,> = latex'}}

% Contradiction
\newcommand{\contradiction}{{\hbox{%
    \setbox0=\hbox{$\mkern-3mu\times\mkern-3mu$}%
    \setbox1=\hbox to0pt{\hss$\times$\hss}%
    \copy0\raisebox{0.5\wd0}{\copy1}\raisebox{-0.5\wd0}{\box1}\box0
}}}

\newtheorem{lemma}[section]{Lemma}

% Sign Charts
\newdimen\tcolw \tcolw=2.5em % the column width
\edef\ecatcode{\catcode`&=\the\catcode`&\relax}\catcode`&=4
\def\sgchart#1#2{\vbox{\offinterlineskip\halign{\hfil##\quad&##\hfil\crcr\sgchartA#2,:,%
   \omit\sgchartR&\kern.2pt\sgchartS{.5\tcolw}\relax\sgchartE#1,\relax,%
   \sgchartS{.5\tcolw}\relax\cr
   \noalign{\kern2pt}&\def~{}\kern.5\tcolw\sgchartD#1,\relax,\cr}}}
\def\sgchartA#1:#2,{\cr\ifx,#1,\else $#1$&\sgchartB#2{}\expandafter\sgchartA\fi}
\def\sgchartB#1{\hbox to\tcolw{\hss$#1$\hss}\sgchartC}
\def\sgchartC#1{\ifx,#1,\else
   \strut\vrule\kern-.4pt\hbox to\tcolw{\hss$#1$\hss}\expandafter\sgchartC\fi}
\def\sgchartD#1#2,{\ifx\relax#1\else\hbox to\tcolw{\hss$#1#2$\hss}\expandafter\sgchartD\fi}
\def\sgchartE#1#2,{\ifx\relax#1\else
    \ifx~#1\sgchartS\tcolw\circ \else\sgchartS\tcolw\bullet\fi \expandafter\sgchartE\fi}
\def\sgchartR{\leaders\vrule height2.8pt depth-2.4pt\hfil}
\def\sgchartS#1#2{\hbox to#1{\kern-.2pt\sgchartR \ifx\relax#2\else
   \kern-.7pt$#2$\kern-.7pt\sgchartR\fi\kern-.2pt}}
\ecatcode
%++++++++++++++++++++++++++++++++++++++++
% title stuff

\makeatletter
\renewcommand{\maketitle}{\bgroup\setlength{\parindent}{0pt}
    \begin{flushleft}
        \textbf{\@title} \\ \vskip0.2cm
        \begingroup
            \fontsize{14pt}{12pt}\selectfont
            \courseNumber: \courseName 
            \vskip0.3cm 
            \professor
        \endgroup \vskip0.3cm
        \dueDate \hfill\rlap{}\bf{\name} \\ \vskip0.1cm
        \hrulefill
    \end{flushleft}\egroup 
}
\makeatother

\title{\Large\bf{\psetName}}

\begin{document}
    \maketitle
    \thispagestyle{plain}
    \section*{Worksheet 4 (Cardinality) Question 14}
    Show that the set of real numbers between 0 and 1 (not including the endpoints) has the same cardinality as the set of real numbers.
    \section*{Answer}
        Let $S = \set*{x \mid 0 < x < 1}$.

        Let $f: \reals \to S$, where $f(x) = \df{1}{\pi}\tan^{-1}x + \df{1}{2}$. 
        \begin{lemma}
            \textbf{$f$ is injective.} An injective function is a function where no two inputs map to the same output. In other words, if $f(x) = f(y)$, then $x = y$. 
            \\[12pt]
            Suppose that $a,b \in \reals$.
            \[f(a) = \df{1}{\pi}\tan^{-1}a + \df{1}{2} \quad f(b) = \df{1}{\pi}\tan^{-1}b + \df{1}{2}\]
            \begin{align*}
                f(a) &= f(b) \\
                \df{1}{\pi}\tan^{-1}a + \df{1}{2} &= \df{1}{\pi}\tan^{-1}b + \df{1}{2} \\
                \tan^{-1}a &= \tan^{-1}b 
                \intertext{Since the range of $\tan^{-1}x$ is $\paren*{-\frac{\pi}{2}, \frac{\pi}{2}}$, we can take the tangent of both sides, as $\tan x$ is defined for all values in the range of $\tan^{-1}x$.}
                \tan\paren*{\tan^{-1}a} &= \tan\paren*{\tan^{-1}b} \\
                a &= b
            \end{align*}
            Since there are no 2 distinct values in the domain $\reals$ that map to the same value in the codomain $S$, $f$ is injective.
            \label{lemma:injective}
        \end{lemma}

        \begin{lemma}
            \textbf{$f$ is surjective.} A surjective function is a function where every element in the codomain is mapped to by at least one element in the domain. In other words, if $y \in S$, then there exists an $x \in \reals$ such that $f(x) = y$.
            \\[12pt]
            Suppose that $y \in S$. Suppose that $x = \tan\paren*{\pi\paren*{y - \df{1}{2}}}$, a value in the domain $\reals$.
            \begin{align*}
                f(x) &= \df{1}{\pi}\tan^{-1}x + \df{1}{2} \\
                f\paren*{\tan\paren*{\pi\paren*{y - \df{1}{2}}}} &= \df{1}{\pi}\tan^{-1}\paren*{\tan\paren*{\pi\paren*{y - \df{1}{2}}}} + \df{1}{2} \\
                &= \df{1}{\pi}\paren*{\pi\paren*{y - \df{1}{2}}} + \df{1}{2} \\
                &= y - \df{1}{2} + \df{1}{2} \\
                &= y
            \end{align*}
            As for all $y \in S$, there exists an $x \in \reals$ such that $f(x) = y$, specifically $x = \tan\paren*{\pi\paren*{y - \df{1}{2}}}$, $f$ is surjective.
            \label{lemma:surjective}
        \end{lemma}

        \begin{lemma}
            \textbf{$f$ is a bijection.} A bijection is a function that is both injective and surjective. 
            \\[12pt]
            Since $f$ is injective from \hyperref[lemma:injective]{Lemma 1} and surjective from \hyperref[lemma:injective]{Lemma 2}, $f$ is a bijection.
            \label{lemma:bijection}
        \end{lemma}

        \begin{theorem}
            Let $S = \set*{x \mid 0 < x < 1}$ and $\reals$ be the set of all real numbers. Then $|S| = |\reals|$.
            \label{thm:cardinality}
        \end{theorem}
        \begin{proof}
            We will prove the theorem, that the cardinality of the set of real numbers between 0 and 1 is the same as the cardinality of $\reals$, by constructing a function that is a bijection between $S$ and $\reals$. This is because two sets have the same cardinality if and only if there is a bijection between them. 

            Let $f: \reals \to S$, where $f(x) = \df{1}{\pi}\tan^{-1}x + \df{1}{2}$. 
        
            The range of $\tan^{-1}x$ is $\paren*{-\df{\pi}{2}, \df{\pi}{2}}$ and by a vertical compression by a factor of $\df{1}{\pi}$, the range becomes $\paren*{-\df{1}{2}, \df{1}{2}}$. We then vertically shift $\df{1}{\pi} \tan^{-1}x$ by $\df{1}{2}$, resulting in a range of $(0, 1)$, our desired codomain for $f$.
            \\[12pt]
            $f$ is a bijection (see \hyperref[lemma:bijection]{Lemma 3}). As a bijection exists between $S$ and $\reals$, their cardinalities are the same---$|S| = |\reals|$.
        \end{proof}
\end{document} 
