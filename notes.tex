%++++++++++++++++++++++++++++++++++++++++
\documentclass[article, 12pt]{article}
\usepackage{float}
\usepackage{setspace}
\usepackage{tabu} % extra features for tabular environment
\usepackage{amsmath}  % improve math presentation
\usepackage{graphicx} % takes care of graphic including machinery
\usepackage[margin=1in]{geometry} % decreases margins
\usepackage{cite} % takes care of citations
\usepackage[final]{hyperref} % adds hyper links inside the generated pdf file
\usepackage{tikz}
\usepackage{caption} 
\usepackage{fancyhdr}
\usepackage{amssymb} % symbols like /therefore
\usepackage{amsthm} % proofs
\usepackage{enumerate} % lettered lists
\usepackage{mathtools} % macros
\usetikzlibrary{scopes}
% \usepackage{xcolor} \pagecolor[rgb]{0.12549019607,0.1294117647,0.13725490196} \color[rgb]{0.82352941176,0.76862745098,0.62745098039} % dark theme
\theoremstyle{definition}
\newtheorem{example}{Example}[subsection]
\newtheorem*{remark}{Remark}
\newtheorem{theorem}{Theorem}[subsection]
\newtheorem{definition}{Definition}[subsection]
\newtheorem{corollary}{Corollary}[subsection]
\hypersetup{
	colorlinks=false,      % false: boxed links; true: colored links
	linkcolor=blue,        % color of internal links
	citecolor=blue,        % color of links to bibliography
	filecolor=magenta,     % color of file links
	urlcolor=blue         
}
\usepackage{physics}
\usepackage{siunitx}
\usepackage{tikz,pgfplots}
\usepackage[outline]{contour} % glow around text
\usetikzlibrary{calc}
\usetikzlibrary{angles,quotes} % for pic
\usetikzlibrary{arrows.meta}
\tikzset{>=latex} % for LaTeX arrow head
\contourlength{1.2pt}

\colorlet{xcol}{blue!70!black}
\colorlet{vcol}{green!60!black}
\colorlet{myred}{red!70!black}
\colorlet{myblue}{blue!70!black}
\colorlet{mygreen}{green!70!black}
\colorlet{mydarkred}{myred!70!black}
\colorlet{mydarkblue}{myblue!60!black}
\colorlet{mydarkgreen}{mygreen!60!black}
\colorlet{acol}{red!50!blue!80!black!80}
\tikzstyle{CM}=[red!40!black,fill=red!80!black!80]
\tikzstyle{xline}=[xcol,thick,smooth]
\tikzstyle{mass}=[line width=0.6,red!30!black,fill=red!40!black!10,rounded corners=1,
                  top color=red!40!black!20,bottom color=red!40!black!10,shading angle=20]
\tikzstyle{faded mass}=[dashed,line width=0.1,red!30!black!40,fill=red!40!black!10,rounded corners=1,
                        top color=red!40!black!10,bottom color=red!40!black!10,shading angle=20]
\tikzstyle{rope}=[brown!70!black,very thick,line cap=round]
\def\rope#1{ \draw[black,line width=1.4] #1; \draw[rope,line width=1.1] #1; }
\tikzstyle{force}=[->,myred,very thick,line cap=round]
\tikzstyle{velocity}=[->,vcol,very thick,line cap=round]
\tikzstyle{Fproj}=[force,myred!40]
\tikzstyle{myarr}=[-{Latex[length=3,width=2]},thin]
\def\tick#1#2{\draw[thick] (#1)++(#2:0.12) --++ (#2-180:0.24)}
\DeclareMathOperator{\sn}{sn}
\DeclareMathOperator{\cn}{cn}
\DeclareMathOperator{\dn}{dn}
\def\N{80} % number of samples in plots


\usepackage{titling}
\renewcommand\maketitlehooka{\null\mbox{}\vfill}
\renewcommand\maketitlehookd{\vfill\null}
\usepackage{siunitx} % units
\usepackage{verbatim} 
\newcommand{\courseNumber}{MATH 1700}
\newcommand{\courseName}{Ideas in Mathematics}
\newcommand{\professor}{Professor Rimmer}
\newcommand{\name}{Denny Cao}
\pagestyle{fancy}
\fancyhf{}% clears all header and footer fields
\fancyfoot[C]{--~\thepage~--}
\renewcommand*{\headrulewidth}{0.4pt}
\renewcommand*{\footrulewidth}{0pt}
\lhead{\name}
\chead{\courseNumber: \courseName}
\rhead{\professor}


\fancypagestyle{plain}{%
  \fancyhf{}% clears all header and footer fields
  \fancyfoot[C]{--~\thepage~--}%
  \renewcommand*{\headrulewidth}{0pt}%
  \renewcommand*{\footrulewidth}{0pt}%
}

% Shortcuts
\DeclarePairedDelimiter\ceil{\lceil}{\rceil} % ceil function
\DeclarePairedDelimiter\floor{\lfloor}{\rfloor} % floor function

\DeclarePairedDelimiter\paren{(}{)} % parenthesis

\newcommand{\df}{\displaystyle\frac} % displaystyle fraction
\newcommand{\qeq}{\overset{?}{=}} % questionable equality

\newcommand{\Mod}[1]{\;\mathrm{mod}\; #1} % modulo operator

% Sets
\DeclarePairedDelimiter\set{\{}{\}}
\newcommand{\unite}{\cup}
\newcommand{\inter}{\cap}

\newcommand{\reals}{\mathbb{R}} % real numbers: textbook is Z^+ and 0
\newcommand{\ints}{\mathbb{Z}}
\newcommand{\nats}{\mathbb{N}}
\newcommand{\rats}{\mathbb{Q}}

\newcommand{\degree}{^\circ}

% Counting
\newcommand\perm[2][^n]{\prescript{#1\mkern-2.5mu}{}P_{#2}}
\newcommand\comb[2][^n]{\prescript{#1\mkern-0.5mu}{}C_{#2}}

\setlength\parindent{0pt}

% Sign Charts
\newdimen\tcolw \tcolw=2.5em % the column width
\edef\ecatcode{\catcode`&=\the\catcode`&\relax}\catcode`&=4
\def\sgchart#1#2{\vbox{\offinterlineskip\halign{\hfil##\quad&##\hfil\crcr\sgchartA#2,:,%
   \omit\sgchartR&\kern.2pt\sgchartS{.5\tcolw}\relax\sgchartE#1,\relax,%
   \sgchartS{.5\tcolw}\relax\cr
   \noalign{\kern2pt}&\def~{}\kern.5\tcolw\sgchartD#1,\relax,\cr}}}
\def\sgchartA#1:#2,{\cr\ifx,#1,\else $#1$&\sgchartB#2{}\expandafter\sgchartA\fi}
\def\sgchartB#1{\hbox to\tcolw{\hss$#1$\hss}\sgchartC}
\def\sgchartC#1{\ifx,#1,\else
   \strut\vrule\kern-.4pt\hbox to\tcolw{\hss$#1$\hss}\expandafter\sgchartC\fi}
\def\sgchartD#1#2,{\ifx\relax#1\else\hbox to\tcolw{\hss$#1#2$\hss}\expandafter\sgchartD\fi}
\def\sgchartE#1#2,{\ifx\relax#1\else
    \ifx~#1\sgchartS\tcolw\circ \else\sgchartS\tcolw\bullet\fi \expandafter\sgchartE\fi}
\def\sgchartR{\leaders\vrule height2.8pt depth-2.4pt\hfil}
\def\sgchartS#1#2{\hbox to#1{\kern-.2pt\sgchartR \ifx\relax#2\else
   \kern-.7pt$#2$\kern-.7pt\sgchartR\fi\kern-.2pt}}
\ecatcode
%++++++++++++++++++++++++++++++++++++++++
\title{
    \vspace{2in}
    \textmd{\textbf{\courseNumber: \courseName}}
    \normalsize\vspace{0.1in}\\
    \vspace{0.1in}\large{\text{\professor}}
    \vspace{3in}
}

\author{\name}
\date{Final: April 26, 2023}

\begin{document}
    \maketitle
    \thispagestyle{empty}
    \pagebreak
    \tableofcontents
    \pagebreak

    \section{Pigeonhole Principle}
    \subsection{Introduction}
    \begin{theorem}
        \textbf{Pigeonhole Principle}: If $k$ is a positive integer and $k+1$ or more objects are placed into $k$ boxes, then there is at least one box containing two or more of the objects.
    \end{theorem}
    \begin{corollary}
        A function $f$ from a set with $k+1$ or more elements to a set with $k$ elements is not one-to-one.
    \end{corollary}
    \subsection{Generalized Pigeonhole Principle}
    \begin{theorem}
        \textbf{Generalized Pigeonhole Principle}: If $N$ objects are placed into $k$ boxes, then there is at least one box containing at least $\ceil*{\df{N}{k}}$ objects.
    \end{theorem}
    Here are some proofs using the pigeonhole principle:
    \begin{example}
        \textbf{Show that for every integer $n$ there is a multiple of $n$ that has only 0s and 1s in its decimal expansion.}
        \begin{proof}
            Let $n$ be a positive integer. Consider the $n+1$ integers $1,11,111,\dots,11\dots 1$ (where the last integer in this list is the integer with $n+1$ 1s in its decimal expansion). Note that there are $n$ possible remainders when an integer is divided by $n$. Because there are $n+1$ integers in this list, by the pigeonhole principle, there must be two with the same remainder when divided by $n$. The larger of these integers less the smaller one is a multiple of $n$, which has decimal expansion with only 0s and 1s.
        \end{proof}
    \end{example}
    \begin{example}
        \textbf{How many cards must be selected from a standard deck of 52 cards to guarentee that:}
        \begin{enumerate}[a)]
            \item at least three cards are of the same suit?
            \item at least three hearts are selected?
        \end{enumerate}
        \begin{proof}
            \textbf{a)} Suppose there are 4 boxes, one for each suit, and as cards are selected they are placed in their respective box. Using the generalized pigeonhole principle, we see that if $N$ cards are selected, there is at least 1 box contaning at least $\ceil*{N/4}$ cards. Thus, we know that at least 3 cards of 1 suit are selected if $\ceil*{N/4} \geq 3$. The smallest integer $N$ to satisfy this inequality is $2\times 4+1=9$, so we must select at least 9 cards to guarentee that at least 3 cards are of the same suit.
        \end{proof}
        \begin{itemize}
            \item Note that if 8 cards are selected, it is possible to have 2 cards of each suit, so more than eight cards are needed.
        \end{itemize} 
    \begin{proof}
        \textbf{b)} We do not use the generalized pigeonhole principle because we want to make sure that there are 3 hearts, not just 3 cards of a suit. Note that in the worst case, we can select all the clubs, diamonds, and spades, 39 cards in all, before we select a single heart. The next 3 cards will all be hearts, so we may need to select 42 cards to get 3 hearts.
    \end{proof}
    \end{example}
    \subsection{Some Elegant Applications of the Pigeonhole Principle}
    \begin{example}
        During a month with 30 days, a baseball team plays at least one game a day, but no more than 45 games. Show that there must be a period of some number of consecutive days during which the team must play exactly 14 games.
        \begin{proof}
            Let $a_j$ be the number of games played on or before the $j$th day of the month. Then $a_1, a_2, \dots, a_{30}$ is an increasing sequence of distinct positive integers, with $1 \leq a_j \leq 45$. Moreover, $a_1 + 14, a_2 + 14, \dots, a_j + 14$ is also an increasing sequence of distinct positive integers, with $15 \leq a_j + 14 \leq 59$. \\

            The 60 positive integers $a_1, a_2, \dots, a_{30}, a_1 + 14, a_2 + 14, \dots, a_j + 14$ are all less than or equal to 59. Hence, by the pigeonhole principle two of these integers must be equal. Because the integers $a_j, j = 1,2,\dots, 30$, are all distinct and the integers $a_j+14, j= 1,2,\dots,30$ are all distinct, there must be indices $i$ and $j$ with $a_i = a_j + 14$. This means that exactly 14 games were played from day $j+1$ to day $i$.
        \end{proof}
    \end{example}
    \begin{example}
        Show that among any $n+1$ positive integers not exceeding $2n$ there must be an integer that divides one of the other integers.
        \begin{proof}
            Write out each of the $n+1$ integers $a_1, a_2, \dots, a_{n+1}$ as a power of 2 times an odd integer. In other words, let $a_j=2^{k_j}q_j$, for $j=1,2,\dots n+1$, where $k_j$ is a nonnegative integer and $q_j$ is odd. The integers $q_1, q_2, \dots, q_{n+1}$ are all odd positive integers less than $2n$. Because there are only $n$ odd positive integers less than $2n$, it follows from the pigeonhole principle that two of the integers $q_1, q_2, \dots, q_{n+1}$ must be equal. Therefore, there are distinct integers $i$ and $j$ such that $q_i = q_j$. Let $q$ be the common value of $q_i$ and $q_j$. Then $a_i = 2^{k_i}q$ and $a_j = 2^{k_j}q$. It follows that if $k_i < k_j$, then $a_i$ divides $a_j$; while if $k_i > k_j$, then $a_j$ divides $a_i$. In either case, there is an integer that divides one of the other integers.
        \end{proof}
    \end{example}
    \begin{theorem}
        Every sequence of $n^2 + 1$ distinct real numbers contains a subsequence of $n+1$ that is either strictly increasing or strictly decreasing.
    \end{theorem}    
    \section{Numbers and Infinity}
    \subsection{Introduction}
    \begin{definition}
        The set of natural numbers is denoted by $\nats$: $\set*{1,2,3, \dots}$. For the purposes of this class, $\nats$ does not include 0.
    \end{definition}
    \begin{definition}
        A function is finitely many if a function can map each element to a subset of $\nats$: $\set*{1,2,3, \dots, n}$.
    \end{definition}
    \subsection{Infinite Sets}
    \begin{definition}
        A set is infinite if a function can map each element to an element of $\nats$. 
    \end{definition}
    \begin{example}
        Prove that the set of even integers is infinite.
        \begin{proof}
            We can separate the set of even integers into two subsets: Positive even integers and negative even integers. Let $f(x)$ be a function that maps each element of the set of even integers to a subset of $\nats$. Let $f(x)$ be defined as follows:
            \[ f(x) = \begin{cases}
                {x} & x \in \set*{2k \mid k \in \nats}  \\
                -2x & x \in \set*{2k - 1 \mid k \in \nats}
               \end{cases} \]
            As each value $x \in \nats$ is mapped to a subset of $\nats$, $f(x)$ is a function.
        \end{proof}
    \end{example}
\end{document}
