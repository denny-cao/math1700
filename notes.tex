%++++++++++++++++++++++++++++++++++++++++
\documentclass[article, 12pt]{article}
\usepackage{float}
\usepackage{setspace}
\usepackage{tabu} % extra features for tabular environment
\usepackage{amsmath}  % improve math presentation
\usepackage{graphicx} % takes care of graphic including machinery
\usepackage[margin=1in]{geometry} % decreases margins
\usepackage{cite} % takes care of citations
\usepackage[final]{hyperref} % adds hyper links inside the generated pdf file
\usepackage{tikz}
\usepackage{caption} 
\usepackage{fancyhdr}
\usepackage{amssymb} % symbols like /therefore
\usepackage{amsthm} % proofs
\usepackage{enumerate} % lettered lists
\usepackage{mathtools} % macros
\usetikzlibrary{scopes}
\usepackage{xcolor} \pagecolor[rgb]{0.12549019607,0.1294117647,0.13725490196} \color[rgb]{0.82352941176,0.76862745098,0.62745098039} % dark theme
\theoremstyle{definition}
\newtheorem{example}{Example}[subsection]
\newtheorem*{remark}{Remark}
\newtheorem{theorem}{Theorem}[subsection]
\newtheorem{definition}{Definition}[subsection]
\newtheorem{corollary}{Corollary}[subsection]
\hypersetup{
	colorlinks=false,      % false: boxed links; true: colored links
	linkcolor=blue,        % color of internal links
	citecolor=blue,        % color of links to bibliography
	filecolor=magenta,     % color of file links
	urlcolor=blue         
}
\usepackage{physics}
\usepackage{siunitx}
\usepackage{tikz,pgfplots}
\usepackage[outline]{contour} % glow around text
\usetikzlibrary{calc}
\usetikzlibrary{angles,quotes} % for pic
\usetikzlibrary{arrows.meta}
\tikzset{>=latex} % for LaTeX arrow head
\contourlength{1.2pt}

\colorlet{xcol}{blue!70!black}
\colorlet{vcol}{green!60!black}
\colorlet{myred}{red!70!black}
\colorlet{myblue}{blue!70!black}
\colorlet{mygreen}{green!70!black}
\colorlet{mydarkred}{myred!70!black}
\colorlet{mydarkblue}{myblue!60!black}
\colorlet{mydarkgreen}{mygreen!60!black}
\colorlet{acol}{red!50!blue!80!black!80}
\tikzstyle{CM}=[red!40!black,fill=red!80!black!80]
\tikzstyle{xline}=[xcol,thick,smooth]
\tikzstyle{mass}=[line width=0.6,red!30!black,fill=red!40!black!10,rounded corners=1,
                  top color=red!40!black!20,bottom color=red!40!black!10,shading angle=20]
\tikzstyle{faded mass}=[dashed,line width=0.1,red!30!black!40,fill=red!40!black!10,rounded corners=1,
                        top color=red!40!black!10,bottom color=red!40!black!10,shading angle=20]
\tikzstyle{rope}=[brown!70!black,very thick,line cap=round]
\def\rope#1{ \draw[black,line width=1.4] #1; \draw[rope,line width=1.1] #1; }
\tikzstyle{force}=[->,myred,very thick,line cap=round]
\tikzstyle{velocity}=[->,vcol,very thick,line cap=round]
\tikzstyle{Fproj}=[force,myred!40]
\tikzstyle{myarr}=[-{Latex[length=3,width=2]},thin]
\def\tick#1#2{\draw[thick] (#1)++(#2:0.12) --++ (#2-180:0.24)}
\DeclareMathOperator{\sn}{sn}
\DeclareMathOperator{\cn}{cn}
\DeclareMathOperator{\dn}{dn}
\def\N{80} % number of samples in plots


\usepackage{titling}
\renewcommand\maketitlehooka{\null\mbox{}\vfill}
\renewcommand\maketitlehookd{\vfill\null}
\usepackage{siunitx} % units
\usepackage{verbatim} 
\newcommand{\courseNumber}{MATH 1700}
\newcommand{\courseName}{Ideas in Mathematics}
\newcommand{\professor}{Professor Rimmer}
\newcommand{\name}{Denny Cao}
\pagestyle{fancy}
\fancyhf{}% clears all header and footer fields
\fancyfoot[C]{--~\thepage~--}
\renewcommand*{\headrulewidth}{0.4pt}
\renewcommand*{\footrulewidth}{0pt}
\lhead{\name}
\chead{\courseNumber: \courseName}
\rhead{\professor}


\fancypagestyle{plain}{%
  \fancyhf{}% clears all header and footer fields
  \fancyfoot[C]{--~\thepage~--}%
  \renewcommand*{\headrulewidth}{0pt}%
  \renewcommand*{\footrulewidth}{0pt}%
}

% Shortcuts
\DeclarePairedDelimiter\ceil{\lceil}{\rceil} % ceil function
\DeclarePairedDelimiter\floor{\lfloor}{\rfloor} % floor function

\DeclarePairedDelimiter\paren{(}{)} % parenthesis

\newcommand{\df}{\displaystyle\frac} % displaystyle fraction
\newcommand{\qeq}{\overset{?}{=}} % questionable equality

\newcommand{\Mod}[1]{\;\mathrm{mod}\; #1} % modulo operator
\newcommand{\Div}[1]{\;\mathrm{div}\; #1} % integer division operator

% Sets
\DeclarePairedDelimiter\set{\{}{\}}
\newcommand{\unite}{\cup}
\newcommand{\inter}{\cap}


\newcommand{\reals}{\mathbb{R}}
\newcommand{\realspos}{\mathbb{R}^+} % real numbers: textbook is Z^+
\newcommand{\ints}{\mathbb{Z}}
\newcommand{\posints}{\mathbb{Z}^+}
\newcommand{\nats}{\mathbb{N}} % textbook is Z^+ and 0
\newcommand{\rats}{\mathbb{Q}}
\newcommand{\comps}{\mathbb{C}}
\newcommand{\irrs}{\mathbb{I}} % Irrational Numbers; Rimmer uses this
\newcommand{\whole}{\mathbb{W}} % Whole Numbers; Rimmer uses this

\newcommand{\powset}{\mathcal{P}} % power set

\newcommand{\degree}{^\circ}

\newcommand{\comp}{\circ} % composition
% Counting
\newcommand\perm[2][^n]{\prescript{#1\mkern-2.5mu}{}P_{#2}}
\newcommand\comb[2][^n]{\prescript{#1\mkern-0.5mu}{}C_{#2}}

% Shortcuts
\newcommand{\xor}{\oplus} % exclusive or
\newcommand{\true}{\textbf{T}} % true
\newcommand{\false}{\textbf{F}} % false
\newcommand{\lra}{\leftrightarrow} % iff

\setlength\parindent{0pt}

% Sign Charts
\newdimen\tcolw \tcolw=2.5em % the column width
\edef\ecatcode{\catcode`&=\the\catcode`&\relax}\catcode`&=4
\def\sgchart#1#2{\vbox{\offinterlineskip\halign{\hfil##\quad&##\hfil\crcr\sgchartA#2,:,%
   \omit\sgchartR&\kern.2pt\sgchartS{.5\tcolw}\relax\sgchartE#1,\relax,%
   \sgchartS{.5\tcolw}\relax\cr
   \noalign{\kern2pt}&\def~{}\kern.5\tcolw\sgchartD#1,\relax,\cr}}}
\def\sgchartA#1:#2,{\cr\ifx,#1,\else $#1$&\sgchartB#2{}\expandafter\sgchartA\fi}
\def\sgchartB#1{\hbox to\tcolw{\hss$#1$\hss}\sgchartC}
\def\sgchartC#1{\ifx,#1,\else
   \strut\vrule\kern-.4pt\hbox to\tcolw{\hss$#1$\hss}\expandafter\sgchartC\fi}
\def\sgchartD#1#2,{\ifx\relax#1\else\hbox to\tcolw{\hss$#1#2$\hss}\expandafter\sgchartD\fi}
\def\sgchartE#1#2,{\ifx\relax#1\else
    \ifx~#1\sgchartS\tcolw\circ \else\sgchartS\tcolw\bullet\fi \expandafter\sgchartE\fi}
\def\sgchartR{\leaders\vrule height2.8pt depth-2.4pt\hfil}
\def\sgchartS#1#2{\hbox to#1{\kern-.2pt\sgchartR \ifx\relax#2\else
   \kern-.7pt$#2$\kern-.7pt\sgchartR\fi\kern-.2pt}}
\ecatcode
%++++++++++++++++++++++++++++++++++++++++
\title{
    \vspace{2in}
    \textmd{\textbf{\courseNumber: \courseName}}
    \normalsize\vspace{0.1in}\\
    \vspace{0.1in}\large{\text{\professor}}
    \vspace{3in}
}

\author{\name}
\date{Final: April 26, 2023}

\begin{document}
    \maketitle
    \thispagestyle{empty}
    \pagebreak
    \tableofcontents
    \pagebreak

    \section{Pigeonhole Principle}
    \subsection{Introduction}
    \begin{theorem}
        \textbf{Pigeonhole Principle}: If $k$ is a positive integer and $k+1$ or more objects are placed into $k$ boxes, then there is at least one box containing two or more of the objects.
    \end{theorem}
    \begin{corollary}
        A function $f$ from a set with $k+1$ or more elements to a set with $k$ elements is not one-to-one.
    \end{corollary}
    \subsection{Generalized Pigeonhole Principle}
    \begin{theorem}
        \textbf{Generalized Pigeonhole Principle}: If $N$ objects are placed into $k$ boxes, then there is at least one box containing at least $\ceil*{\df{N}{k}}$ objects.
    \end{theorem}
    Here are some proofs using the pigeonhole principle:
    \begin{example}
        \textbf{Show that for every integer $n$ there is a multiple of $n$ that has only 0s and 1s in its decimal expansion.}
        \begin{proof}
            Let $n$ be a positive integer. Consider the $n+1$ integers $1,11,111,\dots,11\dots 1$ (where the last integer in this list is the integer with $n+1$ 1s in its decimal expansion). Note that there are $n$ possible remainders when an integer is divided by $n$. Because there are $n+1$ integers in this list, by the pigeonhole principle, there must be two with the same remainder when divided by $n$. The larger of these integers less the smaller one is a multiple of $n$, which has decimal expansion with only 0s and 1s.
        \end{proof}
    \end{example}
    \begin{example}
        \textbf{How many cards must be selected from a standard deck of 52 cards to guarentee that:}
        \begin{enumerate}[a)]
            \item at least three cards are of the same suit?
            \item at least three hearts are selected?
        \end{enumerate}
        \begin{proof}
            \textbf{a)} Suppose there are 4 boxes, one for each suit, and as cards are selected they are placed in their respective box. Using the generalized pigeonhole principle, we see that if $N$ cards are selected, there is at least 1 box contaning at least $\ceil*{N/4}$ cards. Thus, we know that at least 3 cards of 1 suit are selected if $\ceil*{N/4} \geq 3$. The smallest integer $N$ to satisfy this inequality is $2\times 4+1=9$, so we must select at least 9 cards to guarentee that at least 3 cards are of the same suit.
        \end{proof}
        \begin{itemize}
            \item Note that if 8 cards are selected, it is possible to have 2 cards of each suit, so more than eight cards are needed.
        \end{itemize} 
    \begin{proof}
        \textbf{b)} We do not use the generalized pigeonhole principle because we want to make sure that there are 3 hearts, not just 3 cards of a suit. Note that in the worst case, we can select all the clubs, diamonds, and spades, 39 cards in all, before we select a single heart. The next 3 cards will all be hearts, so we may need to select 42 cards to get 3 hearts.
    \end{proof}
    \end{example}
    \subsection{Some Elegant Applications of the Pigeonhole Principle}
    \begin{example}
        During a month with 30 days, a baseball team plays at least one game a day, but no more than 45 games. Show that there must be a period of some number of consecutive days during which the team must play exactly 14 games.
        \begin{proof}
            Let $a_j$ be the number of games played on or before the $j$th day of the month. Then $a_1, a_2, \dots, a_{30}$ is an increasing sequence of distinct positive integers, with $1 \leq a_j \leq 45$. Moreover, $a_1 + 14, a_2 + 14, \dots, a_j + 14$ is also an increasing sequence of distinct positive integers, with $15 \leq a_j + 14 \leq 59$. \\

            The 60 positive integers $a_1, a_2, \dots, a_{30}, a_1 + 14, a_2 + 14, \dots, a_j + 14$ are all less than or equal to 59. Hence, by the pigeonhole principle two of these integers must be equal. Because the integers $a_j, j = 1,2,\dots, 30$, are all distinct and the integers $a_j+14, j= 1,2,\dots,30$ are all distinct, there must be indices $i$ and $j$ with $a_i = a_j + 14$. This means that exactly 14 games were played from day $j+1$ to day $i$.
        \end{proof}
    \end{example}
    \begin{example}
        Show that among any $n+1$ positive integers not exceeding $2n$ there must be an integer that divides one of the other integers.
        \begin{proof}
            Write out each of the $n+1$ integers $a_1, a_2, \dots, a_{n+1}$ as a power of 2 times an odd integer. In other words, let $a_j=2^{k_j}q_j$, for $j=1,2,\dots n+1$, where $k_j$ is a nonnegative integer and $q_j$ is odd. The integers $q_1, q_2, \dots, q_{n+1}$ are all odd positive integers less than $2n$. Because there are only $n$ odd positive integers less than $2n$, it follows from the pigeonhole principle that two of the integers $q_1, q_2, \dots, q_{n+1}$ must be equal. Therefore, there are distinct integers $i$ and $j$ such that $q_i = q_j$. Let $q$ be the common value of $q_i$ and $q_j$. Then $a_i = 2^{k_i}q$ and $a_j = 2^{k_j}q$. It follows that if $k_i < k_j$, then $a_i$ divides $a_j$; while if $k_i > k_j$, then $a_j$ divides $a_i$. In either case, there is an integer that divides one of the other integers.
        \end{proof}
    \end{example}
    \begin{theorem}
        Every sequence of $n^2 + 1$ distinct real numbers contains a subsequence of $n+1$ that is either strictly increasing or strictly decreasing.
    \end{theorem}    
    \section{Numbers and Infinity}
    \subsection{Introduction}
    \begin{definition}
        The set of natural numbers is denoted by $\nats$: $\set*{1,2,3, \dots}$. For the purposes of this class, $\nats$ does not include 0.
    \end{definition}
    \begin{definition}
        A function is finitely many if a function can map each element to a subset of $\nats$: $\set*{1,2,3, \dots, n}$.
    \end{definition}
    \subsection{Infinite Sets}
    \begin{definition}
        A set is infinite if a function can map each element to an element of $\nats$. 
    \end{definition}
    \begin{example}\label{ex:infinite}
        Prove that the set of even integers is infinite.
        \begin{proof}
            We can separate the set of even integers into two subsets: Positive even integers and negative even integers. Let $f(x)$ be a function that maps each element of the set of even integers to a subset of $\nats$. Let $f(x)$ be defined as follows:
            \[ f(x) = \begin{cases}
                {x} & x \in \set*{2k \mid k \in \nats}  \\
                -2x & x \in \set*{2k - 1 \mid k \in \nats}
               \end{cases} \]
            As each value $x \in \nats$ is mapped to a subset of $\nats$, $f(x)$ is a function.
        \end{proof}
    \end{example}
    \subsection{Machine Method}
    \begin{definition}
        The \textbf{machine method} is a method to demonstrate that a set is infinite by building a machine (an algorithm) that takes as input a finite list and names an element of the set that is not in the list. \textbf{The machine shows that no finite list can contain all the elements of the set.}

    \end{definition}
    \begin{itemize}
        \item The machine method can be used to show that a collection is infinite because if we run the machine forever on repeat, it will produce an infinite list.
        \item A better explanation: If a collection were finite, it would be possible to include all its elements in a finite list. The machine method demonstrates that no finite list can contain the entire collection. So the collection cannot be finite.
    \end{itemize}
    \begin{example}
        Use the machine method to show that there are infinitely many prime numbers.
        \begin{proof}
            We give our machine a rule: Take the greatest number in the list and add 2. As we only fed our machine even numbers, adding 2 will always give an even number. Because we added 2 to the greatest number on our list, the sum will be greater than that number. Since our output is greater than the greatest number on the finite list, it must be greater than every number on the finite list. Therefore, the output was not part of our original list.
        \end{proof}
    \end{example}
    \subsection{Prime Numbers}
    \begin{definition}
        $x$ \textbf{is a factor of} $y$ if $y=kx, k \in \nats$. Another interpretation is that $x \Div{y}$.
    \end{definition}
    \begin{definition}
        A natural number is a \textbf{prime number} if it is greater than 1 and has no factors other than 1 and itself.
    \end{definition}
    \begin{definition}
        A natural number is a \textbf{composite number} if it is greater than 1 and has at least one factor other than 1 and itself.
    \end{definition}
    \begin{itemize}
        \item From \hyperref[ex:infinite]{the Exercise}, we know that the set of even integers is infinite. Thus, it follows that there are infinitely many composite numbers.
        \item The number 1 is \textbf{neither prime nor composite}.
    \end{itemize}
    \section{Sets}
    \subsection{Introduction}
    \begin{definition}
        A \textbf{set} is an unordered collection of distinct objects called \textbf{elements} or \textbf{members} of the set. A set is said to \textbf{contain} its elements. We write $a \in A$ to denote that $a$ is an element of the set $A$. The notation $a \notin A$ denotes that $a$ is not an element of the set $A$.   
    \end{definition}

    Sets of types of numbers:
    \begin{itemize}
        \item Natural Numbers: $\nats = \{0, 1, 2, 3, \dots\} = \{ \posints \unite 0\}$
        \item Integers: $\ints = \{\dots, -2, -1, 0, 1, 2, \dots\}$
        \item Positive Integers: $\ints^+ = \{1, 2, 3, \dots\}$
        \item Rational Numbers: $\rats = \set*{\df{a}{b} \mid a, b \in \ints \text{ and } b \neq 0}$
        \item Real Numbers: $\reals$
        \item Positive Real Numbers: $\reals^+$
        \item Complex Numbers: $\comps$
    \end{itemize}

    \begin{definition}
        \textbf{Equality of Sets}:
        \begin{equation*}
            A=B \lra \forall x (x \in A \lra x \in B) \lra A \subseteq B \land B \subseteq A   
        \end{equation*}
    \end{definition}

    \begin{definition}
        \textbf{Empty Set}: $\emptyset = \set*{}$
    \end{definition}
    \stepcounter{subsection}
    \subsection{Subsets}
    \begin{definition}
        \textbf{Subset}:
        \begin{equation*}
            A \subseteq B \lra \forall x (x \in A \lra x \in B) \lra B \supseteq A
        \end{equation*}
        To show that $A \not\subseteq B$, show $\exists x (x \in A \land x \not\in B)$.
    \end{definition}

    \begin{theorem}
        For every set $S$, $\emptyset \subseteq S$ and $S \subseteq S$.
    \end{theorem}
    \subsection{Size of a Set}
    \begin{definition}
        Let $S$ be a set. If there are exactly $n$ distinct elements in $S$, where $n$ is a nonnegative integer, we say that $S$ is a \textbf{finite set} and that $n$ is the \textbf{cardinality} of $S$, denoted by $|S|$.
        \begin{itemize}
            \item Note: Theorem 2.1.3.1!
        \end{itemize}
    \end{definition}
    \subsection{Power Sets}
    \begin{definition}
        Let $S$ be a set. The \textbf{power set} of $S$, denoted by $\powset(S)$, is the set of all subsets of $S$.
    \end{definition}
    \begin{theorem} Cardinality of a power set
        \begin{equation*}
            |\powset(S)| = 2^{|S|}
        \end{equation*}
    \end{theorem}
    \subsection{Cartesian Products}
    \begin{definition}
        Let $A$ and $B$ be sets. The \textbf{Cartesian product} of $A$ and $B$, denoted by $A \times B$, is the set of all ordered pairs $(a,b)$ where $a \in A$ and $b \in B$. Hence:
        \begin{equation*}
            A \times B = \{(a,b) \mid a \in A \land b \in B\}    
        \end{equation*}
    \end{definition}
    \subsection{Set Operations}
    \subsection{Introduction}
    \begin{definition}
        Let $A$ and $B$ be sets. The \textbf{union} of the sets $A$ and $B$, denoted $A \unite B$, is the set that contains those elements that are in either $A$ or $B$ or both. Hence:
        \begin{equation*}
            A \unite B = \{x \mid x \in A \lor x \in B\}
        \end{equation*}
    \end{definition}
    \begin{definition}
        Let $A$ and $B$ be sets. The \textbf{intersection} of the sets $A$ and $B$, denoted $A \inter B$, is the set that contains those elements in both $A$ and $B$. Hence:
        \begin{equation*}
            A \inter B = \{x \mid x \in A \land x \in B\}
        \end{equation*}
    \end{definition}
    \begin{definition}
        Two sets are called \textbf{disjoint} if their intersection is the emptyset.
    \end{definition}
    \begin{definition}
        Let $A$ and $B$ be sets. The \textbf{difference} of the sets $A$ and $B$, denoted $A - B$, is the set that contains those elements in $A$ but not in $B$. It is also called the \textbf{complement of $B$ with respect to $A$}. Hence:
        \begin{equation*}
            A - B = \{x \mid x \in A \land x \not\in B\}
        \end{equation*}
    \end{definition}
    \begin{definition}
        Let $U$ be the universal set. The \textbf{complement} of a set $A$, denoted $\overline{A}$, is the set $U - A$. Hence:
        \begin{equation*}
            \overline{A} = \{x \mid x \in U \land x \not\in A\}
        \end{equation*}
    \end{definition}
    \begin{definition}
        Let $A$ and $B$ be sets. The \textbf{symmetric difference} of $A$ and $B$ is the set of elements that are in either $A$ or $B$ but not in both. It is denoted by $A \xor B$. Hence:
        \begin{equation*}
            A \xor B = (A \unite B) - (A \inter B)
        \end{equation*} 
    \end{definition}
    \subsection{Set Identities}
    \begin{figure}[H]
        \centering
        {\renewcommand{\arraystretch}{1.5}
        \begin{tabular}{|l|c|}
            \hline
            Identity & Name \\
            \hline
            $A \inter U = A$ & Identity Laws \\
            $A \unite \emptyset = A$ & \\
            \hline
            $A \unite U = U$ & Domination Laws \\
            $A \inter \emptyset = \emptyset$ & \\
            \hline
            $A \unite A = A$ & Idempotent Laws \\
            $A \inter A = A$ & \\
            \hline
            $\overline{(\overline{A})} = A$ & Complementation Law \\
            \hline
            $A \unite B = B \unite A$ & Commutative Laws \\
            $A \inter B = B \inter A$ & \\
            \hline
            $A \unite (B \unite C) = (A \unite B) \unite C$ & Associative Laws \\
            $A \inter (B \inter C) = (A \inter B) \inter C$ & \\
            \hline
            $A \unite (B \inter C) = (A \unite B) \inter (A \unite C)$ & Distributive Laws \\
            $A \inter (B \unite C) = (A \inter B) \unite (A \inter C)$ & \\
            \hline
            $\overline{A \inter B} = \overline{A} \unite \overline{B}$ & De Morgan's Laws \\
            $\overline{A \unite B} = \overline{A} \inter \overline{B}$ & \\
            \hline
            $A \unite (A \inter B) = A$ & Absorption Laws \\
            $A \inter (A \unite B) = A$ & \\
            \hline
            $A \unite \overline{A} = U$ & Complement Laws \\
            $A \inter \overline{A} = \emptyset$ & \\
            \hline
        \end{tabular}}
        \caption{Set Identities}
    \end{figure}
    There are 3 ways to prove that two sets are equal:
    \begin{enumerate}
        \item Showing that they are subsets of each other. (Definition 2.2)
        \item Membership tables.
        \item Set identities.
    \end{enumerate}
    A \textbf{membership table} considers each combination of the atomic sets (the original sets used to produce the sets on each side) that an element can belong to and verify that elements in the same combinations of sets belong to both the sets in the identity. Use a 1 to indicate that an element belongs to a set and a 0 to indicate that it does not. For example, consider the following identity:
    \begin{equation*}
        A \unite (A \inter B) = A
    \end{equation*}
    We can construct a membership table for this identity as follows:
    \begin{center}
        \begin{tabular}{|c|c|c|}
            \hline
            $A$ & $B$ & $A \unite (A \inter B)$ \\
            \hline
            1 & 1 & 1 \\
            1 & 0 & 1 \\
            0 & 1 & 0 \\
            0 & 0 & 0 \\
            \hline
        \end{tabular}
    \end{center}
    Since the columns are the same, we can conclude that the sets are equal.
    \subsection{Generalized Unions and Intersections}
    \begin{definition}
        The \textbf{union} of a collection of sets is the set that contains those elements that are members of at least one set in the collection. It is denoted by:
        \begin{equation*}
            A_1 \unite A_2 \unite \cdots A_n = \bigcup_{i=1}^{n} A_i
        \end{equation*}
    \end{definition}
    \begin{definition}
        The \textbf{intersection} of a collection of sets is the set that contains those elements that are members of all sets in the collection. It is denoted by:
        \begin{equation*}
            A_1 \inter A_2 \inter \cdots A_n = \bigcap_{i=1}^{n} A_i
        \end{equation*}
    \end{definition}
    \section{Functions}
    \subsection{Introduction}
    \begin{definition}
        Let $A$ and $B$ be nonempty sets. A \textbf{function} $f$ from $A$ to $B$ is an assignment of exactly one element of $B$ to each element of $A$. We write $f(a) = b$ if $b$ is the unique element of $B$ assigned by the function $f$ to the element $a$ of $A$. If $f$ is a function from $A$ to $B$, we write $f: A \to B$.
        \begin{itemize}
            \item Functions are sometimes also called \textbf{mappings} or \textbf{transformations}
        \end{itemize}
    \end{definition}
    \begin{definition}
        Let $f: A \to B$ be a function. $A$ is the \textbf{domain} of $f$ and $B$ is the \textbf{codomain} of $f$. If $f(a) = b$, we say that $b$ is the \textbf{image} of $a$ and $a$ is the \textbf{preimage} of $b$. The \textbf{range}, or \textbf{image} of $f$ is the set of all images of elements of $A$. Also, if $f$ is a function from $A$ to $B$, we say that $f$ \textbf{maps} $A$ to $B$.
        \begin{itemize}
            \item Codomain is set of possible values of the function and range is the set of all elements of the codomain that are achieved as the value of $f$ for at least one element of the domain.
            \item Two functions are \textbf{equal} when they have the same domain, same codomain, and map each each element of their common domain to the same element in their common codomain.
        \end{itemize}
    \end{definition}
    \begin{definition}
        Let $f_1$ and $f_2$ be functions from $A$ to $B$. Then $f_1 + f_2$ and $f_1f_2$ are also functions from $A$ to $B$ defined $\forall x \in A$ by:
        \begin{align*}
            (f_1 + f_2)(x) &= f_1(x) + f_2(x) \\
               (f_1f_2)(x) &= f_1(x)f_2(x)
        \end{align*}
    \end{definition}
    \begin{definition}
        Let $f$ be a function from $A$ to $B$ and let $S \subseteq A$. The \textbf{image} of $S$ under the function $f$ is the subset of $B$ that consists of the images of the elements of $S$. We denote the image of $S$ by $f(S)$, so:
        \begin{equation*}
            f(S) = \set*{t \mid \exists s \in S (t = f(s))} = \set*{f(s) \mid s \in S}
        \end{equation*}
    \end{definition}
    \subsection{One-to-One and Onto Functions}
    \begin{definition}
        A function $f$ with domain $A$ is \textbf{one-to-one} if and only if: 
        \begin{equation*}
            \forall a \forall b (a,b \in A \land (f(a) = f(b) \to a = b))
        \end{equation*}
        \begin{itemize}
            \item A function $f$ is one-to-one if and only if $f(a) \neq f(b)$ whenever $a \neq b$. This is obtained by taking the contrapositve of the implication in the definition.
        \end{itemize}
    \end{definition}
    \begin{definition}
        A function $f$ whose domain $A$ and codomain $B$ are subsets of the set of real numbers is called \textbf{increasing} if $f(x) \leq f(y)$ whenever $x < y$ and $x,y \in A$. Hence:
        \begin{equation*}
            \forall x \forall y (x,y \in A \land x < y \to f(x) \leq f(y))
        \end{equation*}
    \end{definition}
    \begin{definition}
        A function $f$ whose domain $A$ and codomain $B$ are subsets of the set of real numbers is called \textbf{strictly increasing} if $f(x) < f(y)$ whenever $x < y$ and $x,y \in A$. Hence:
        \begin{equation*}
            \forall x \forall y (x,y \in A \land x < y \to f(x) < f(y))
        \end{equation*}
    \end{definition}
    \begin{definition}
        A function $f$ whose domain $A$ and codomain $B$ are subsets of the set of real numbers is called \textbf{decreasing} if $f(x) \geq f(y)$ whenever $x < y$ and $x,y \in A$. Hence:
        \begin{equation*}
            \forall x \forall y (x,y \in A \land x < y \to f(x) \geq f(y))
        \end{equation*}
    \end{definition}
    \begin{definition}
        A function $f$ whose domain $A$ and codomain $B$ are subsets of the set of real numbers is called \textbf{strictly decreasing} if $f(x) > f(y)$ whenever $x < y$ and $x,y \in A$. Hence:
        \begin{equation*}
            \forall x \forall y (x,y \in A \land x < y \to f(x) > f(y))
        \end{equation*}
    \end{definition}
    \begin{definition}
        A function $f$ from $A$ to $B$ is \textbf{onto}, or a \textbf{surjection}, if and only if for every element $y \in B$ there exists an element $x \in A$ such that $f(x) = y$. Hence:
        \begin{equation*}
            \forall y \exists x (f(x) = y)
        \end{equation*}
        where the domain for $x$ is $A$ and the domain of $y$ is $B$.
        \begin{itemize}
            \item $f$ is \textbf{surjective} if it is onto.
        \end{itemize}
    \end{definition}
    \begin{definition}
        The function $f$ is a \textbf{one-to-one correspondence} if it is both one-to-one and onto. 
        \begin{itemize}
            \item Such a function is \textbf{bijective}
        \end{itemize}
        \label{def:one-to-one-correspondence}
    \end{definition}
    \begin{figure}[H]
        \centering
        {\renewcommand{\arraystretch}{1.5}
        \begin{tabular}{|l p{25em}|}
            \hline
            \multicolumn{2}{|l|}{Suppose that $f: A \to B$.} \\
            \hline
            Show $f$ is injective: & Show that if $f(x) = f(y)$ for arbitrary $x,y \in A$, then $x = y$ \\
            Show $f$ is not injective: & Find particular elements, $x,y \in A$ such that $x \neq y$ and $f(x)=f(y)$. \\
            Show $f$ is surjective: & Consider an arbitrary element $y \in B$ and find an element $x \in A$ such that $f(x) = y$. \\
            Show $f$ is not surjective: & Find a particular $y \in B$ such that $f(x) \neq y$ for all $x \in A$. \\
            Show $f$ is bijective: & Show that $f$ is both injective and surjective. \\
            \hline
        \end{tabular}}
    \end{figure}
    \subsection{Inverse Functions and Composite Functions}
    \begin{definition}
        Let $f$ be a one-to-one correspondence from the set $A$ to the set $B$. The \textbf{inverse function} of $f$ is denoted by $f^{-1}$: Hence: 
        \begin{equation*}
            f^{-1}(b) = a \text{ when } f(a) = b
        \end{equation*}
        \begin{itemize}
            \item A one-to-one correspondence $f$ is \textbf{invertible} because we can define an inverse function $f^{-1}$.
            \item A function is \textbf{invertible} if it is not a one-to-one correspondence, because the inverse of $f$ does not exist.
        \end{itemize}
    \end{definition}
    \begin{definition}
        Let $g$ be a function from the set $A$ to the set $B$ and let $f$ be a function from the set $B$ to the set $C$. The \textbf{composition} of the functions $f$ and $g$, denoted for all $a \in A$ by $f \comp g$, is the function from $A$ to $C$ defined by:
        \begin{equation*}
            (f \comp g)(a) = f(g(a))
        \end{equation*}
        \begin{itemize}
            \item $f \comp g$ assigns the element $a$ of $A$ the element assigned by $f$ to $g(a)$. 
            \item The domain of $f \comp g$ is the domain of $g$.
            \item The range of $f \comp g$ is the image of the range of $g$ with respect to $f$.
            \item The composition $f \comp g$ cannot be defined unless the range of $g$ is a subset of the domain of $f$. 
            \item \textbf{Not Commutative!} 
            \begin{equation*}
                f \comp g \neq g \comp f
            \end{equation*}
            \item When composing with inverse function, an identity function is obtained:
            \begin{equation*}
                f \comp f^{-1}(a) = f^{-1} \comp f(a) = a
            \end{equation*}
        \end{itemize}
    \end{definition}
    \stepcounter{subsection}
    \subsection{Some Important Functions}
    \begin{definition}
        The \textbf{floor function} assigns to the real number $x$ the largest integer that is less than or equal to $x$. The value of the floor function at $x$ is denoted by $\floor*{x}$. The \textbf{ceiling function} assigns to the real number $x$ the smllest integer that is greater than or equal to $x$. The value of the ceiling function at $x$ is denoted by $\ceil*{x}$.
    \end{definition}
    \begin{figure}[H]
        \centering
        {\renewcommand{\arraystretch}{1.3}
        \begin{tabular}{|c|}
            \hline
            \textbf{$n$ is an integer, $x$ is a real number} \\
            \hline
            $\floor*{x} = n \lra n \leq x < n+1$ \\
            $\ceil*{x} = n \lra n-1 < x \leq n$ \\
            $\floor*{x} = n \lra x-1 < n \leq x$ \\
            $\ceil*{x} = n \lra x \leq n < x+1$ \\
            \hline
            $x-1 < \floor*{x} \leq x \leq \ceil*{x} < x+1$ \\
            \hline
            $\floor*{-x} = -\ceil*{x}$ \\
            $\ceil*{-x} = -\floor*{x}$ \\
            \hline
            $\floor*{x+n} = \floor*{x} + n$ \\
            $\ceil*{x+n} = \ceil*{x} + n$ \\
            \hline
        \end{tabular}}
        \caption{Useful Properties of the Floor and Ceiling Functions}
    \end{figure}
\end{document}
