%++++++++++++++++++++++++++++++++++++++++
\documentclass[article, 12pt]{article}
\usepackage{float}
\usepackage{setspace}
\usepackage{tabu} % extra features for tabular environment
\usepackage{amsmath}  % improve math presentation
\usepackage{graphicx} % takes care of graphic including machinery
\usepackage[margin=1in]{geometry} % decreases margins
\usepackage{cite} % takes care of citations
\usepackage[final]{hyperref} % adds hyper links inside the generated pdf file
\usepackage{tikz}
\usepackage{caption} 
\usepackage{fancyhdr}
\usepackage{amssymb} % symbols like /therefore
\usepackage{amsthm} % proofs
\usepackage{enumerate} % lettered lists
\usepackage{mathtools} % macros
\usepackage[ all]{xy} % for diagrams
\usepackage[inline]{enumitem}
\usepackage{tkz-graph}

\usetikzlibrary{scopes}
% \usepackage{xcolor} \pagecolor[rgb]{0.12549019607,0.1294117647,0.13725490196} \color[rgb]{0.82352941176,0.76862745098,0.62745098039} % dark theme
\theoremstyle{definition}
\newtheorem{example}{Example}[subsubsection]
\newtheorem*{remark}{Remark}
\newtheorem{theorem}{Theorem}[subsubsection]
\newtheorem{definition}{Definition}[subsubsection]
\newtheorem{corollary}{Corollary}[subsubsection]
\hypersetup{
	colorlinks=false,      % false: boxed links; true: colored links
	linkcolor=blue,        % color of internal links
	citecolor=blue,        % color of links to bibliography
	filecolor=magenta,     % color of file links
	urlcolor=blue         
}
\usepackage{physics}
\usepackage{siunitx}
\usepackage{tikz,pgfplots}
\usepackage[outline]{contour} % glow around text
\usetikzlibrary{calc}
\usetikzlibrary{angles,quotes} % for pic
\usetikzlibrary{arrows.meta}
\tikzset{>=latex} % for LaTeX arrow head
\contourlength{1.2pt}

\colorlet{xcol}{blue!70!black}
\colorlet{vcol}{green!60!black}
\colorlet{myred}{red!70!black}
\colorlet{myblue}{blue!70!black}
\colorlet{mygreen}{green!70!black}
\colorlet{mydarkred}{myred!70!black}
\colorlet{mydarkblue}{myblue!60!black}
\colorlet{mydarkgreen}{mygreen!60!black}
\colorlet{acol}{red!50!blue!80!black!80}
\tikzstyle{CM}=[red!40!black,fill=red!80!black!80]
\tikzstyle{xline}=[xcol,thick,smooth]
\tikzstyle{mass}=[line width=0.6,red!30!black,fill=red!40!black!10,rounded corners=1,
                  top color=red!40!black!20,bottom color=red!40!black!10,shading angle=20]
\tikzstyle{faded mass}=[dashed,line width=0.1,red!30!black!40,fill=red!40!black!10,rounded corners=1,
                        top color=red!40!black!10,bottom color=red!40!black!10,shading angle=20]
\tikzstyle{rope}=[brown!70!black,very thick,line cap=round]
\def\rope#1{ \draw[black,line width=1.4] #1; \draw[rope,line width=1.1] #1; }
\tikzstyle{force}=[->,myred,very thick,line cap=round]
\tikzstyle{velocity}=[->,vcol,very thick,line cap=round]
\tikzstyle{Fproj}=[force,myred!40]
\tikzstyle{myarr}=[-{Latex[length=3,width=2]},thin]
\def\tick#1#2{\draw[thick] (#1)++(#2:0.12) --++ (#2-180:0.24)}
\DeclareMathOperator{\sn}{sn}
\DeclareMathOperator{\cn}{cn}
\DeclareMathOperator{\dn}{dn}
\def\N{80} % number of samples in plots


\usepackage{titling}
\renewcommand\maketitlehooka{\null\mbox{}\vfill}
\renewcommand\maketitlehookd{\vfill\null}
\usepackage{siunitx} % units
\usepackage{verbatim} 
\newcommand{\courseNumber}{MATH 1700}
\newcommand{\courseName}{Ideas in Mathematics}
\newcommand{\professor}{Professor Rimmer}
\newcommand{\psetName}{Worksheet 4: Cardinality First Submission}
\newcommand{\dueDate}{Due: February 15, 2023}
\newcommand{\name}{Denny Cao}
\pagestyle{fancy}
\fancyhf{}% clears all header and footer fields
\fancyfoot[C]{--~\thepage~--}
\renewcommand*{\headrulewidth}{0.4pt}
\renewcommand*{\footrulewidth}{0pt}
\lhead{\name}
\chead{\courseNumber: \courseName}
\rhead{\professor}

% new theorem for questions and answers

\newtheorem{question}{Question}

\newtheorem{answer}{Answer}

\fancypagestyle{plain}{%
  \fancyhf{}% clears all header and footer fields
  \fancyfoot[C]{--~\thepage~--}%
  \renewcommand*{\headrulewidth}{0pt}%
  \renewcommand*{\footrulewidth}{0pt}%
}

% Shortcuts
\DeclarePairedDelimiter\ceil{\lceil}{\rceil} % ceil function
\DeclarePairedDelimiter\floor{\lfloor}{\rfloor} % floor function

\DeclarePairedDelimiter\paren{(}{)} % parenthesis

\newcommand{\df}{\displaystyle\frac} % displaystyle fraction
\newcommand{\qeq}{\overset{?}{=}} % questionable equality

\newcommand{\Mod}[1]{\;\mathrm{mod}\; #1} % modulo operator

\newcommand{\comp}{\circ} % composition

% Sets
\DeclarePairedDelimiter\set{\{}{\}}
\newcommand{\unite}{\cup}
\newcommand{\inter}{\cap}

\newcommand{\reals}{\mathbb{R}} % real numbers: textbook is Z^+ and 0
\newcommand{\ints}{\mathbb{Z}}
\newcommand{\nats}{\mathbb{N}}
\newcommand{\rats}{\mathbb{Q}}

\newcommand{\degree}{^\circ}

% Counting
\newcommand\perm[2][^n]{\prescript{#1\mkern-2.5mu}{}P_{#2}}
\newcommand\comb[2][^n]{\prescript{#1\mkern-0.5mu}{}C_{#2}}

% Relations
\newcommand{\rel}{\mathcal{R}} % relation

\setlength\parindent{0pt}

% Directed Graphs
\usetikzlibrary{arrows}
\tikzset{vertex/.style = {shape=circle,draw, minimum size=1.5em,
inner sep=0pt, outer sep=0pt}}
\tikzset{edge/.style = {->,> = latex'}}

% Sign Charts
\newdimen\tcolw \tcolw=2.5em % the column width
\edef\ecatcode{\catcode`&=\the\catcode`&\relax}\catcode`&=4
\def\sgchart#1#2{\vbox{\offinterlineskip\halign{\hfil##\quad&##\hfil\crcr\sgchartA#2,:,%
   \omit\sgchartR&\kern.2pt\sgchartS{.5\tcolw}\relax\sgchartE#1,\relax,%
   \sgchartS{.5\tcolw}\relax\cr
   \noalign{\kern2pt}&\def~{}\kern.5\tcolw\sgchartD#1,\relax,\cr}}}
\def\sgchartA#1:#2,{\cr\ifx,#1,\else $#1$&\sgchartB#2{}\expandafter\sgchartA\fi}
\def\sgchartB#1{\hbox to\tcolw{\hss$#1$\hss}\sgchartC}
\def\sgchartC#1{\ifx,#1,\else
   \strut\vrule\kern-.4pt\hbox to\tcolw{\hss$#1$\hss}\expandafter\sgchartC\fi}
\def\sgchartD#1#2,{\ifx\relax#1\else\hbox to\tcolw{\hss$#1#2$\hss}\expandafter\sgchartD\fi}
\def\sgchartE#1#2,{\ifx\relax#1\else
    \ifx~#1\sgchartS\tcolw\circ \else\sgchartS\tcolw\bullet\fi \expandafter\sgchartE\fi}
\def\sgchartR{\leaders\vrule height2.8pt depth-2.4pt\hfil}
\def\sgchartS#1#2{\hbox to#1{\kern-.2pt\sgchartR \ifx\relax#2\else
   \kern-.7pt$#2$\kern-.7pt\sgchartR\fi\kern-.2pt}}
\ecatcode
%++++++++++++++++++++++++++++++++++++++++
% title stuff

\makeatletter
\renewcommand{\maketitle}{\bgroup\setlength{\parindent}{0pt}
    \begin{flushleft}
        \textbf{\@title} \\ \vskip0.2cm
        \begingroup
            \fontsize{14pt}{12pt}\selectfont
            \courseNumber: \courseName 
            \vskip0.3cm 
            \professor
        \endgroup \vskip0.3cm
        \@date \hfill\rlap{}\bf{\name} \\ \vskip0.1cm
        \hrulefill
    \end{flushleft}\egroup 
}
\makeatother

\title{\LARGE\bf{\psetName}}
\author{\name}
\date{\dueDate}

\author{\name}
\date{\dueDate}

\begin{document}
    \maketitle
    \thispagestyle{empty}

    \section{Warm-Up Problems}
    \begin{question}
        State what it means for sets $A$ and $B$ to have the same cardinality.
    \end{question}

    \begin{answer}
        Sets $A$ and $B$ have the same cardinality if and only if there exists a bijection between $A$ and $B$.
    \end{answer}

    \begin{question}
        State what it means for a set $A$ to be \textit{countable}.
    \end{question}

    \begin{answer}
        A set $A$ is countable if and only if it is finite or if $A$ and $\nats$ have the same cardinality---there exists a bijection between $A$ and the set of natural numbers.
    \end{answer}

    \section{Some Differences Between Finite Sets and Infinite Sets}
    \begin{question}
        Give an example of a function between two \textit{infinite} sets with the same cardinality which is injective but not surjective.
    \end{question}

    \begin{answer}
        Let $f: \nats \to \ints$ be defined as 
        \[ f(x) = \begin{cases} 
            x & x \in \set*{2k \mid k \in \nats} \\
            -x & x \in \set*{2k - 1 \mid k \in \nats}
        \end{cases} \]
        $f$ is injective because every element of $\ints^+$ is mapped to a unique element of $\ints$. However, $f$ is not surjective, as $0 \in \ints$ but does not have a preimage in $\nats$.
    \end{answer}

    \begin{question}
        Give an example of a function between two \textit{infinite} sets with the same cardinality which is surjective but not injective.
    \end{question}

    \begin{answer}
        Let $f: \ints \to \ints$ be defined as:
        \[ f(x) = x^3 - x \]
        $f$ is surjective. Let $y \in \ints$. $\forall y \exists x \in \ints$ such that $f(x) = y$, $x^3 - x = y$. As there exists a preimage for every element of $\ints$, $f$ is surjective. However, $f$ is not injective, as $f(0) = 0$ and $f(1) = 0$.
    \end{answer}

    \begin{question}
        If we replaced the word “infinite” with the word “finite” in each of the two previous problems, would either problem have a solution? Why or why not?    
    \end{question}

    \begin{answer}
        Question 3 would not have an answer, but Question 4 would have an answer. In Question 3, if two sets $A$ and $B$ are finite and $|A| = |B|$, then the two sets have the same amount of elements. To maintain injectiveness from a mapping from $A \to B$, each element in $A$ must map to a unique element in $B$. It follows that this would create a bijection, which is not a surjection. 
        \\[12pt]
        Question 4 would have an answer, as it is possible to map all values in $B$ to the same value in $A$. This would create a surjection, but not an injection.
    \end{answer}

    \begin{question}
        Give an example of a function between two \textit{infinite} sets with the same cardinality which is neither injective nor surjective.    
    \end{question}

    \begin{answer}
        Let $f: \ints \to \ints$ be defined as $f(x) = 1$. 
        \\[12pt]
        $f$ is not injective, as $f(0) = f(1) = 1$. $f$ is not surjective, as not all elements of $\ints$ has a preimage.
    \end{answer}

    \begin{question}
        Give an example of a function between two \textit{finite} sets with the same cardinality which is neither injective nor surjective.    
    \end{question}

    \begin{answer}
        Let $f: \set*{1, 2, 3} \to \set*{1, 2, 3}$ be defined as $f(x) = 1$. 
        \\[12pt]
        $f$ is not injective, as $f(1) = f(2) = f(3) = 1$. $f$ is not surjective, as not all elements of $\set*{1, 2, 3}$ has a preimage.
    \end{answer}

    \begin{question}
        Why don't your examples in this section contradict the definition of ``same cardinality?''  
    \end{question}

    \begin{answer}
        If two sets $A$ and $B$ have the same cardinality, it is possible to create a bijection between $A$ and $B$. For every example that is not bijective, it is still possible to create a bijection between $A$ and $B$ using a different function $f$.
    \end{answer}

    \section{Sets That Have the Same Cardinality}
    \begin{question}
        Construct a bijective function between the set of odd natural numbers and the set of even natural numbers. Do the two sets have the same cardinality?
    \end{question}

    \begin{answer}
        Let $f: \set*{2k - 1 \mid k \in \nats} \to \set*{2k \mid k \in \nats}$ be defined as $f(x) = x + 1$.
        \begin{proof}
            $f$ is injective. Let $a,b \in \set*{2k - 1 \mid k \in \nats}$. $f(a) = f(b) \to a + 1 = b + 1 \to a = b$. Thus, $f(a) = f(b) \leftrightarrow a = b$.
            \\[12pt]
            $f$ is surjective. Let $y \in \set*{2k \mid k \in \nats}$. $\exists x \in \set*{2k - 1 \mid k \in \nats}$ such that $f(x) = y$, specifically $x = y-1$. As there exists a preimage for every element of $\set*{2k \mid k \in \nats}$, $f$ is surjective. 
            \\[12pt]
            $f$ is bijective, as it is both injective and surjective. The two sets have the same cardinality, as there exists a bijection between the two sets.
        \end{proof}
    \end{answer}

    \begin{question}
        Show that the set of natural numbers that have 3 as a factor is countable.
    \end{question}
    
    \begin{answer} \
        \begin{proof}
            Let the set of natural numbers that have 3 as a factor be $A = \set*{3k \mid k \in \nats}$. To prove that $A$ is countable, we must show that there exists a bijection between $\nats$ and $A$. Let $f: \nats \to A$ be defined as $f(x) = 3x$. 
            \\[12pt]
            $f$ is injective. Let $a,b \in \nats$. $f(a) = f(b) \to 3a = 3b \to a = b$. Thus, $f(a) = f(b) \leftrightarrow a = b$.
            \\[12pt]
            $f$ is surjective. Let $y \in A$. $\exists x \in \nats$ such that $f(x) = y$, specifically $x = \displaystyle\frac{y}{3}$. As there exists a preimage for every element of $A$, $f$ is surjective.
            \\[12pt]
            $f$ is bijective, as it is both injective and surjective. The two sets have the same cardinality, as there exists a bijection between the two sets. Therefore, $A$ is countable.
        \end{proof}
    \end{answer}

    \begin{question}
        Exhibit a bijective function between the set of prime numbers and the set of integers. You do \textit{not} need to represent your function algebraically.
    \end{question}

    \begin{answer}
        We can list the prime numbers in ascending order: 2, 3, 5, 7, 11, 13, 17, 19, 23, 29, 31, $\dots$
        \\[12pt]
        Then, we can assign each prime number a natural number in the following way:
        \begin{align*}
            2 & \mapsto 1 \\
            3 & \mapsto 2 \\
            5 & \mapsto 3 \\
            7 & \mapsto 4 \\
            11 & \mapsto 5 \\
            13 & \mapsto 6 \\
            17 & \mapsto 7 \\
            19 & \mapsto 8 \\
            23 & \mapsto 9 \\
            29 & \mapsto 10 \\
            31 & \mapsto 11 \\
               &\shortvdotswithin{\mapsto}
        \end{align*}
        This function is bijective because each prime number is assigned a unique natural number and vice versa. 
    \end{answer}

    \begin{question}
        Show that the set [0, 1] has the same cardinality as the set [0, 1000]. (The notation [a, b] means the set of all real numbers at least a and no greater than b, including the numbers a and b.)    
    \end{question}  

    \begin{answer} \
        \begin{proof}
            Let $f: [0, 1] \to [0, 1000]$ be defined as $f(x) = 1000x$. 
            \\[12pt]
            $f$ is injective. Let $a,b \in [0, 1]$. $f(a) = f(b) \to 1000a = 1000b \to a = b$. Thus, $f(a) = f(b) \leftrightarrow a = b$.
            \\[12pt]
            $f$ is surjective. Let $y \in [0, 1000]$. $\exists x \in [0, 1]$ such that $f(x) = y$, specifically $x = \displaystyle\frac{y}{1000}$. As there exists a preimage for every element of $[0, 1000]$, $f$ is surjective.
            \\[12pt]
            $f$ is bijective, as it is both injective and surjective. The two sets have the same cardinality, as there exists a bijection between the two sets.
        \end{proof}
    \end{answer}

    \begin{question}
        You come across a previously undiscovered ancient language. The language appears to have four letters, which can be used in any order to form words. There is no known limit on the number of letters per word, although each individual word may only have finitely many letters. Explain why the set of all possible words in this language is countable.
    \end{question}

    \begin{answer} \
        \begin{proof}
            Let each letter be represented by a natural number. For example, let one letter be represented by 1, another by 2, another by 3, and another by 4. Then, we can represent any word by a sequence of natural numbers. 
            \\[12pt]
            Then, each natural number represents a unique word, and each word is represented by a unique natural number. Therefore, there exists a bijection between the set of all possible words and the set of all natural numbers. Thus, the set of all possible words is countable.
        \end{proof}
    \end{answer}
    \section{Reflection}
    \textbf{What content do I need to review before attempting the worksheet again? Are there any videos I need to rewatch?}

    I do not feel that I need to review any content before attempting the worksheet again. I feel that I understand the concepts well enough to attempt the worksheet again.
    \\[12pt]
    \textbf{What questions would I like to ask my group during the next class discussion?}

    I would like to ask my group about Question 13, and how I can create a function to map natural numbers to the words. I explained vaguely by saying that each word can be represented by a unique natural number, but I wonder if there is a more concrete way to do this.
\end{document} 
