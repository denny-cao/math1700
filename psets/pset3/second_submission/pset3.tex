%++++++++++++++++++++++++++++++++++++++++
\documentclass[article, 12pt]{article}
\usepackage{float}
\usepackage{setspace}
\usepackage{tabu} % extra features for tabular environment
\usepackage{amsmath}  % improve math presentation
\usepackage{graphicx} % takes care of graphic including machinery
\usepackage[margin=1in]{geometry} % decreases margins
\usepackage{cite} % takes care of citations
\usepackage[final]{hyperref} % adds hyper links inside the generated pdf file
\usepackage{tikz}
\usepackage{caption} 
\usepackage{fancyhdr}
\usepackage{amssymb} % symbols like /therefore
\usepackage{amsthm} % proofs
\usepackage{enumerate} % lettered lists
\usepackage{mathtools} % macros

\usetikzlibrary{scopes}
% \usepackage{xcolor} \pagecolor[rgb]{0.12549019607,0.1294117647,0.13725490196} \color[rgb]{0.82352941176,0.76862745098,0.62745098039} % dark theme
\theoremstyle{definition}
\newtheorem{example}{Example}[subsubsection]
\newtheorem*{remark}{Remark}
\newtheorem{theorem}{Theorem}[subsubsection]
\newtheorem{definition}{Definition}[subsubsection]
\newtheorem{corollary}{Corollary}[subsubsection]
\hypersetup{
	colorlinks=false,      % false: boxed links; true: colored links
	linkcolor=blue,        % color of internal links
	citecolor=blue,        % color of links to bibliography
	filecolor=magenta,     % color of file links
	urlcolor=blue         
}
\usepackage{physics}
\usepackage{siunitx}
\usepackage{tikz,pgfplots}
\usepackage[outline]{contour} % glow around text
\usetikzlibrary{calc}
\usetikzlibrary{angles,quotes} % for pic
\usetikzlibrary{arrows.meta}
\tikzset{>=latex} % for LaTeX arrow head
\contourlength{1.2pt}

\colorlet{xcol}{blue!70!black}
\colorlet{vcol}{green!60!black}
\colorlet{myred}{red!70!black}
\colorlet{myblue}{blue!70!black}
\colorlet{mygreen}{green!70!black}
\colorlet{mydarkred}{myred!70!black}
\colorlet{mydarkblue}{myblue!60!black}
\colorlet{mydarkgreen}{mygreen!60!black}
\colorlet{acol}{red!50!blue!80!black!80}
\tikzstyle{CM}=[red!40!black,fill=red!80!black!80]
\tikzstyle{xline}=[xcol,thick,smooth]
\tikzstyle{mass}=[line width=0.6,red!30!black,fill=red!40!black!10,rounded corners=1,
                  top color=red!40!black!20,bottom color=red!40!black!10,shading angle=20]
\tikzstyle{faded mass}=[dashed,line width=0.1,red!30!black!40,fill=red!40!black!10,rounded corners=1,
                        top color=red!40!black!10,bottom color=red!40!black!10,shading angle=20]
\tikzstyle{rope}=[brown!70!black,very thick,line cap=round]
\def\rope#1{ \draw[black,line width=1.4] #1; \draw[rope,line width=1.1] #1; }
\tikzstyle{force}=[->,myred,very thick,line cap=round]
\tikzstyle{velocity}=[->,vcol,very thick,line cap=round]
\tikzstyle{Fproj}=[force,myred!40]
\tikzstyle{myarr}=[-{Latex[length=3,width=2]},thin]
\def\tick#1#2{\draw[thick] (#1)++(#2:0.12) --++ (#2-180:0.24)}
\DeclareMathOperator{\sn}{sn}
\DeclareMathOperator{\cn}{cn}
\DeclareMathOperator{\dn}{dn}
\def\N{80} % number of samples in plots


\usepackage{titling}
\renewcommand\maketitlehooka{\null\mbox{}\vfill}
\renewcommand\maketitlehookd{\vfill\null}
\usepackage{siunitx} % units
\usepackage{verbatim} 
\newcommand{\courseNumber}{MATH 1700}
\newcommand{\courseName}{Ideas in Mathematics}
\newcommand{\professor}{Professor Rimmer}
\newcommand{\psetName}{Worksheet 3: Sets and Functions Second Submission}
\newcommand{\dueDate}{Due: February 6, 2023}
\newcommand{\name}{Denny Cao}
\pagestyle{fancy}
\fancyhf{}% clears all header and footer fields
\fancyfoot[C]{--~\thepage~--}
\renewcommand*{\headrulewidth}{0.4pt}
\renewcommand*{\footrulewidth}{0pt}
\lhead{\name}
\chead{\courseNumber: \courseName}
\rhead{\professor}


\fancypagestyle{plain}{%
  \fancyhf{}% clears all header and footer fields
  \fancyfoot[C]{--~\thepage~--}%
  \renewcommand*{\headrulewidth}{0pt}%
  \renewcommand*{\footrulewidth}{0pt}%
}

% Shortcuts
\DeclarePairedDelimiter\ceil{\lceil}{\rceil} % ceil function
\DeclarePairedDelimiter\floor{\lfloor}{\rfloor} % floor function

\DeclarePairedDelimiter\paren{(}{)} % parenthesis

\newcommand{\df}{\displaystyle\frac} % displaystyle fraction
\newcommand{\qeq}{\overset{?}{=}} % questionable equality

\newcommand{\Mod}[1]{\;\mathrm{mod}\; #1} % modulo operator

% Sets
\DeclarePairedDelimiter\set{\{}{\}}
\newcommand{\unite}{\cup}
\newcommand{\inter}{\cap}

\newcommand{\reals}{\mathbb{R}} % real numbers: textbook is Z^+ and 0
\newcommand{\ints}{\mathbb{Z}}
\newcommand{\nats}{\mathbb{N}}
\newcommand{\rats}{\mathbb{Q}}

\newcommand{\degree}{^\circ}

% Counting
\newcommand\perm[2][^n]{\prescript{#1\mkern-2.5mu}{}P_{#2}}
\newcommand\comb[2][^n]{\prescript{#1\mkern-0.5mu}{}C_{#2}}

\setlength\parindent{0pt}

% Card Symbols
\DeclareSymbolFont{extraup}{U}{zavm}{m}{n}
\DeclareMathSymbol{\varheart}{\mathalpha}{extraup}{86}
\DeclareMathSymbol{\vardiamond}{\mathalpha}{extraup}{87}

\newcommand{\clubs}{\clubsuit}
\newcommand{\diamonds}{\vardiamond}
\newcommand{\hearts}{\varheart}
\newcommand{\spades}{\spadesuit}

% Contradiction Symbol smashtimes
\newcommand\contradiction{\mathbin{\mathpalette\xhash\relax}}
\newcommand{\xhash}[2]{\ooalign{%
  $#1\xxhash{#1}{-45}$\cr
  $#1\xxhash{#1}{45}$\cr
  }%
}
\newcommand{\xxhash}[2]{\rotatebox[origin=c]{#2}{$#1\parallel$}}
%++++++++++++++++++++++++++++++++++++++++
\title{
    \vspace{2in}
    \textmd{\textbf{\courseNumber: \courseName}}
    \normalsize\vspace{0.1in}\\
    \vspace{0.1in}\Large{\text{\psetName}} \\
    \vspace{0.1in}\large{\text{\professor}}
    \vspace{3in}
}

\author{\name}
\date{\dueDate}

\begin{document}
    \maketitle
    \thispagestyle{empty}
    \pagebreak

    \section{Function Review}
    \begin{enumerate}[(1)] \label{q:1}
        \item \textbf{Consider the relationships below. None of them are functions. For each example, explain why.}
        \begin{enumerate}[(a)]
            \item $f: \set*{a, b, c} \to \nats$ 
            \begin{figure}[H]
                \centering
                \begin{tabular}{c|c}
                    $x$ & $f(x)$ \\
                    \hline
                    a & 12 \\
                    b & 7 \\
                    b & 4 \\
                    c & 128
                \end{tabular}
            \end{figure}
            \begin{proof}
                The definition of a function is that all elements in the domain have a unique image in the codomain. In this case, $b$ maps to two elements in the codomain, $7$ and $4$, and therefore $f$ is not a function.
            \end{proof}
            \item $g: \nats \to \nats , g(n) = n-4$
            \begin{proof}
                The definition of a function is that all elements in the domain have a unique image in the codomain. In this case, 1,2,3, and 4 map to numbers not in the codomain of $\nats$. Therefore, for these values, there is no unique image for all values in the domain, meaning $g$ is not a function.
            \end{proof}
            \item $h: \ints \to \rats, h(z) = \df{1}{z}$
            \begin{proof}
                The definition of a function is that all elements in the domain have a unique image in the codomain. In this case, the preimage of 0 maps to an undefined value, and therefore there is no unique image for all values in the domain, meaning $h$ is not a function.
            \end{proof}
        \end{enumerate}
        \item \textbf{Consider the function from $\nats$ to $\ints$ that multiplies every number in the source by 2. Is this function injective? Is it surjective?}
        \begin{proof}
            \textbf{The function is injective}. The definition of injective for a function $f$ with domain $A$ is \[ \forall x \forall y((x,y \in A \land f(x) = f(y)) \to x = y) \] In this case, $f(x) = 2x$ and $f(y) = 2y$, so if $f(x) = f(y)$, then $2x = 2y \to x = y$. Therefore, the function is injective.
        \end{proof}
        \begin{proof}
            \textbf{The function is not surjective}. The definition of surjective for a function $f$ with domain $A$ is \[ \forall y \exists x(x \in A \land f(x) = y) \] In this case, $f(x) = 2x$. This means that all values in the domain map to an even number. Since the codomain is $\ints$, there are infinitely many integers that are not even---there are infinitely many values in the codomain which do not have a preimage. Therefore, the function is not surjective.
        \end{proof}
        \item \textbf{Consider the function from $\nats$ to $\nats$ that adds one to every number in the source. Is this function injective? Surjective?}
        \begin{proof}
            \textbf{The function is injective}. The definition of injective for a function $f$ with domain $A$ is \[ \forall x \forall y((x,y \in A \land f(x) = f(y)) \to x = y) \] In this case, $f(x) = x+1$ and $f(y) = y+1$, so if $f(x) = f(y)$, then $x+1 = y+1 \to x = y$. Therefore, the function is injective.
        \end{proof}
        \begin{proof}
            \textbf{The function is not surjective}. The definition of surjective for a function $f$ with domain $A$ is \[ \forall y \exists x(x \in A \land f(x) = y) \] In this case, $f(x) = x+1$. To obtain an image of 1, the preimage must be 0. However, 0 is not in the domain of the function, $\nats$, meaning $\exists y \forall x (x \in A \land f(x) \neq y)$. Therefore, the function is not surjective.
        \end{proof}
        \item \textbf{Consider the function from $\ints$ to $\ints$ that adds one to every number in the source. Is this function injective? Surjective?}
        \begin{proof}
            \textbf{The function is injective}. The definition of injective for a function $f$ with domain $A$ is \[ \forall x \forall y((x,y \in A \land f(x) = f(y)) \to x = y) \] In this case, $f(x) = x+1$ and $f(y) = y+1$, so if $f(x) = f(y)$, then $x+1 = y+1 \to x = y$. Therefore, the function is injective.
        \end{proof}
        \begin{proof}
            \textbf{The function is surjective}. The definition of surjective for a function $f$ with domain $A$ is \[ \forall y \exists x(x \in A \land f(x) = y) \] In this case, $f(x) = x+1$. For all elements in the codomain, there exists a preimage in the domain, $x = y-1$. Therefore, the function is surjective.
        \end{proof}
        \item \textbf{Consider the function from $\ints$ to $\reals$ that sends ever integer to itself. Is this function injective? Surjective?}
        \begin{proof}
            \textbf{The function is injective}. The definition of injective for a function $f$ with domain $A$ is \[ \forall x \forall y((x,y \in A \land f(x) = f(y)) \to x = y) \] In this case, $x \in \ints \to f(x) = x$. Let $a,b \in \ints$. As $f(a) = a$ and $f(b) = b$, $f(a) = f(b) \to a = b$. Therefore, the function is injective.
        \end{proof}
        \begin{proof}
            \textbf{The function is not surjective}. The definition of surjective for a function $f$ with domain $A$ is \[ \forall y \exists x(x \in A \land f(x) = y) \] As the domain is $\ints$, the domain maps solely to integers in the codomain. However, the codomain is $\reals$, which contains infinitely many real numbers that are not integers. As $\exists y \forall x (x \in A \land f(x) \neq y)$, the function is not surjective.
        \end{proof}
        \item \textbf{Consider the function from $\nats$ to $\nats$ that subtracts one from every number other than 1, and sends 1 to 1. Is this function injective? Surjective?}
        \begin{proof}
            \textbf{The function is injective}. The definition of injective for a function $f$ with domain $A$ is \[ \forall x \forall y((x,y \in A \land f(x) = f(y)) \to x = y) \]
            We can represent the function as a piecewise function:
            \[ f: \nats \to \nats, f(x) = \begin{cases} 1 & x = 1 \\ x-1 & x \neq 1 \end{cases} \]
            The union of the domains of the two pieces is the domain of the function, $\nats$. The intersection of the domains of the two pieces is the empty set, so the function is injective.
        \end{proof}
        \begin{proof}
            \textbf{The function is surjective}. The definition of surjective for a function $f$ with domain $A$ is
            \[ \forall y \exists x (x \in A \land f(x) = y) \]
            When $x \neq 1$, $f(x) = x - 1$. For all values $y$ in the codomain, there is a value $x$ in the domain, $y+1$, that maps to it. When $x=1$, $y=1$. Despite the image 1 having 2 preimages, 1, and 2, the definition of surjective is still satisfied.
        \end{proof}
        \item \begin{enumerate}[(a)]
            \item \textbf{Write down a function with source $\nats$ and target $\{\spades, \hearts, \diamonds, \clubs\}$ which is not surjective.}
            \[ f: \nats \to \set*{\spades, \hearts, \diamonds, \clubs}, f(x) = \spades \]
            \item \textbf{Are there any injective functions from $\nats$ to $\{\spades, \hearts, \diamonds, \clubs\}$? Give an example or very briefly explain why not.}
            \begin{proof}
                \textbf{There are no injective functions from $\nats$ to $\set*{\spades, \hearts, \diamonds, \clubs}$}. We prove this by the pigeonhole principle. Let the values in the codomain be holes, and let the values in the domain be pigeons. There are 4 holes. With 4 pigeons, we can place each pigeon in each hole. With 5 pigeons, there must be a hole that contains at least 2---with 5 pigeons, the function is not injective. As the cardinality of $\nats > 5$, there will be more pigeons than holes, meaning at least 1 hole contains more than 1 pigeons. Thus, there is no possible injective function.
            \end{proof}
            \item \textbf{Are there any injective functions from $\{\spades, \hearts, \diamonds, \clubs\}$ to $\nats$ which are not surjective? Give an example or very briefly explain why not.}
            \begin{proof}
                \textbf{There are injective functions from $\set*{\spades, \hearts, \diamonds, \clubs}$ to $\nats$ which are not surjective}. As the cardinality of $\nats$ is greater than the cardinality of $\set*{\spades, \hearts, \diamonds, \clubs}$, there are more elements in the codomain than the domain. Thus, it is possible for the elements in the domain to map to unique elements in the codomain. However, since the cardinality of the codomain is greater than the domain, there are elements in the codomain that are not mapped to by the domain. Thus, the function is not surjective.
            \end{proof}
            \end{enumerate}
        \end{enumerate}
    \section{Infinite Sets}
        \begin{enumerate}[(1)]
            \setcounter{enumi}{7}
            \item \textbf{Suppose you have a very large fish bowl, and infinitely many fish nearby. The fish are numbered 1, 2, 3, 4, . . . When you start out, none of the fish are in the bowl. Then suddenly fish \#1 leaps into the bowl. Next, fish \#2, \#3,. . . ,\#10 all jump into the bowl, and fish \#1 jumps back out. After that, fish \#11 through \#100 all jump in, and \#2 jumps out. Next \#101 through \#1000 all jump in and \#3 jumps out. The process continues, ad infinitum.}
            \begin{enumerate}[(a)]
                \item \textbf{How many fish are in the bowl after 4 steps of the above process? 10 steps? 100 steps?}
                
                The number of fish in the bowl after $n$ steps is $f(n)$. We can represent this as a piecewise function:
                \[ f(n) = \begin{cases}
                    0 & n = 0 \\
                    1 - n + 10^{n-1} & n \geq 1
                \end{cases} \]
                The latter case is obtained by the following reasoning. Each iteration, $9(10)^{i-1}$ fish jump in, and 1 fish jumps out. Thus, the number of fish in the bowl after $n$ steps is:
                \begin{align*}
                    1 + \sum_{i=1}^n \paren*{9(10)^{i-1} - 1} &= 1 - n + 9 \sum_{i=1}^n 10^{i-1} \\
                    &= 1 - n + 9 \paren*{\frac{10^n - 1}{9}} \\
                    &= 1 - n + 10^{n-1} 
                \end{align*}
                After 4 steps, $n=4$, there will be 997 fish in the bowl. After 10 steps, $n=10$, there will be 999,999,991 fish in the bowl. After 100 steps, there will be $10^{99} - 99$ fish in the bowl.
                \item \textbf{After infinitely many steps, how many fish are in the bowl? (Hint: At the end of the minute, each fish is either in the bowl or outside the bowl. Which fish are in and which are out?)}
            
                There will be $10^{n-1} - n + 1$ fish in the bowl after $n$ steps. As $n$ approaches infinity, the number of fish in the bowl approaches infinity, since more fish are added ($10^{n-1}$) to the bowl than are removed ($n$).
            \end{enumerate}
    \end{enumerate}
    \section{Russell's Paradox}
        \begin{enumerate}[(1)]
            \setcounter{enumi}{8}
            \item \textbf{Consider the set of all sets that contain themselves. Can you tell whether or not this set contains itself?}
            
            It is impossible to determine; both possibilities are possible.
            \begin{proof}
                Let the set of all sets that contain themselves be $S = \set*{A \mid A \in A}$, where $A$ is a set. There are two possibilities: either $S$ contains itself or it does not. 
                
                If $S$ contains itself, then $S \in S$ by the definition of $S$; $S \in S \to S \in S$. 
                
                If $S$ does not contain itself, then $S \not\in S$ by the definition of $S$; $S \not\in S \to S \not\in S$. 
                
                As both implications are true, it is impossible to determine whether or not $S$ contains itself.
            \end{proof}
            \item \textbf{Now consider the set of all sets that do not contain themselves. Can this set contain itself? Can it not?}
            
            This set cannot contain itself nor not contain itself, as such a set cannot exist.
            \begin{proof}
                By contradiction. Assume for purposes of contradiction that there is a set of all sets that do not contain themselves, $S = \set*{A \mid A \not\in A}$, where $A$ is a set. There are two possibilities: $S$ contains itself or $S$ does not contain itself.
                
                If $S \in S$, then $S \not\in S$, as $S$ is the set of all sets that do not contain themselves; $S \in S \to S \not\in S$. $\contradiction$

                If $S \not\in S$, then $S \in S$, as $S$ is the set of all sets that do not contain themselves ; $S \not\in S \to S \in S$.  $\contradiction$
                
                We reach a paradox, as $S \in S \to S \not\in S$ and $S \not\in S \to S \in S$. Both implications are contradictions, leading to the conclusion that such a set $S$ cannot exist.
            \end{proof}
        \end{enumerate}
        \section{Reflection}
        \textbf{Identify at least one wrong or failed idea that turned out to be helpful or enlightening in some way. For instance, that idea might have helped you solve a problem, or it may have been the start of a conversation that improved your understanding more generally. You can list one of your own ideas, or an idea that originated with a classmate. (Please give your classmate credit!)}

        For \hyperref[q:1]{Question 1}, I had an entire discussion with a friend outside of class, Daniel, from my MATH 163: Discrete Mathematics class at CCP, about what it means to be a function. Initially, I used injective to describe functions, but once I got to part B, I realized that this was wrong, as $g: \nats \to \nats, g(n)=n-4$ is surjective. Daniel messaged me that it is not a function because ``not every element in the domain is `assigned' an element in the codomain''. 
        \\[12pt]
        I then thought that this meant a function must be injective, and we went to a discussion about what it means for a function to be injective. Injective doesn't have any qualifications about how many elements in the domain are mapped to the codomain; it only says that if $f(x_1) = f(x_2) \to x_1 = x_2$, or that every element in the codomain can only be mapped from one element. 
        \\[12pt]
        To be a function, each element in the domain must map to one element in the codomain. Injectivity says nothing about if $f(x_1) \lor f(x_2)$ does not exist. We realized that this is why functions such as $\sqrt{x}$ are functions, as the domain is restricted to $\reals^+ \unite \set*{0}$, and the codomain is restricted to $\reals^+$, and every element in the domain maps to one element in the codomain.

\end{document}
