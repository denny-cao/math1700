%++++++++++++++++++++++++++++++++++++++++
\documentclass[article, 12pt]{article}
\usepackage{float}
\usepackage{setspace}
\usepackage{tabu} % extra features for tabular environment
\usepackage{amsmath}  % improve math presentation
\usepackage{graphicx} % takes care of graphic including machinery
\usepackage[margin=1in]{geometry} % decreases margins
\usepackage{cite} % takes care of citations
\usepackage[final]{hyperref} % adds hyper links inside the generated pdf file
\usepackage{tikz}
\usepackage{caption} 
\usepackage{fancyhdr}
\usepackage{amssymb} % symbols like /therefore
\usepackage{amsthm} % proofs
\usepackage{enumerate} % lettered lists
\usepackage{mathtools} % macros
\usepackage{fdsymbol} % card suit symbols
\usetikzlibrary{scopes}
% \usepackage{xcolor} \pagecolor[rgb]{0.12549019607,0.1294117647,0.13725490196} \color[rgb]{0.82352941176,0.76862745098,0.62745098039} % dark theme
\theoremstyle{definition}
\newtheorem{example}{Example}[subsubsection]
\newtheorem*{remark}{Remark}
\newtheorem{theorem}{Theorem}[subsubsection]
\newtheorem{definition}{Definition}[subsubsection]
\newtheorem{corollary}{Corollary}[subsubsection]
\hypersetup{
	colorlinks=false,      % false: boxed links; true: colored links
	linkcolor=blue,        % color of internal links
	citecolor=blue,        % color of links to bibliography
	filecolor=magenta,     % color of file links
	urlcolor=blue         
}
\usepackage{physics}
\usepackage{siunitx}
\usepackage{tikz,pgfplots}
\usepackage[outline]{contour} % glow around text
\usetikzlibrary{calc}
\usetikzlibrary{angles,quotes} % for pic
\usetikzlibrary{arrows.meta}
\tikzset{>=latex} % for LaTeX arrow head
\contourlength{1.2pt}

\colorlet{xcol}{blue!70!black}
\colorlet{vcol}{green!60!black}
\colorlet{myred}{red!70!black}
\colorlet{myblue}{blue!70!black}
\colorlet{mygreen}{green!70!black}
\colorlet{mydarkred}{myred!70!black}
\colorlet{mydarkblue}{myblue!60!black}
\colorlet{mydarkgreen}{mygreen!60!black}
\colorlet{acol}{red!50!blue!80!black!80}
\tikzstyle{CM}=[red!40!black,fill=red!80!black!80]
\tikzstyle{xline}=[xcol,thick,smooth]
\tikzstyle{mass}=[line width=0.6,red!30!black,fill=red!40!black!10,rounded corners=1,
                  top color=red!40!black!20,bottom color=red!40!black!10,shading angle=20]
\tikzstyle{faded mass}=[dashed,line width=0.1,red!30!black!40,fill=red!40!black!10,rounded corners=1,
                        top color=red!40!black!10,bottom color=red!40!black!10,shading angle=20]
\tikzstyle{rope}=[brown!70!black,very thick,line cap=round]
\def\rope#1{ \draw[black,line width=1.4] #1; \draw[rope,line width=1.1] #1; }
\tikzstyle{force}=[->,myred,very thick,line cap=round]
\tikzstyle{velocity}=[->,vcol,very thick,line cap=round]
\tikzstyle{Fproj}=[force,myred!40]
\tikzstyle{myarr}=[-{Latex[length=3,width=2]},thin]
\def\tick#1#2{\draw[thick] (#1)++(#2:0.12) --++ (#2-180:0.24)}
\DeclareMathOperator{\sn}{sn}
\DeclareMathOperator{\cn}{cn}
\DeclareMathOperator{\dn}{dn}
\def\N{80} % number of samples in plots


\usepackage{titling}
\renewcommand\maketitlehooka{\null\mbox{}\vfill}
\renewcommand\maketitlehookd{\vfill\null}
\usepackage{siunitx} % units
\usepackage{verbatim} 
\newcommand{\courseNumber}{MATH 1700}
\newcommand{\courseName}{Ideas in Mathematics}
\newcommand{\professor}{Professor Rimmer}
\newcommand{\psetName}{Worksheet 3: Sets and Functions First Submission}
\newcommand{\dueDate}{Due: February 1, 2023}
\newcommand{\name}{Denny Cao}
\pagestyle{fancy}
\fancyhf{}% clears all header and footer fields
\fancyfoot[C]{--~\thepage~--}
\renewcommand*{\headrulewidth}{0.4pt}
\renewcommand*{\footrulewidth}{0pt}
\lhead{\name}
\chead{\courseNumber: \courseName}
\rhead{\professor}


\fancypagestyle{plain}{%
  \fancyhf{}% clears all header and footer fields
  \fancyfoot[C]{--~\thepage~--}%
  \renewcommand*{\headrulewidth}{0pt}%
  \renewcommand*{\footrulewidth}{0pt}%
}

% Shortcuts
\DeclarePairedDelimiter\ceil{\lceil}{\rceil} % ceil function
\DeclarePairedDelimiter\floor{\lfloor}{\rfloor} % floor function

\DeclarePairedDelimiter\paren{(}{)} % parenthesis

\newcommand{\df}{\displaystyle\frac} % displaystyle fraction
\newcommand{\qeq}{\overset{?}{=}} % questionable equality

\newcommand{\Mod}[1]{\;\mathrm{mod}\; #1} % modulo operator

% Sets
\DeclarePairedDelimiter\set{\{}{\}}
\newcommand{\unite}{\cup}
\newcommand{\inter}{\cap}

\newcommand{\reals}{\mathbb{R}} % real numbers: textbook is Z^+ and 0
\newcommand{\ints}{\mathbb{Z}}
\newcommand{\nats}{\mathbb{N}}
\newcommand{\rats}{\mathbb{Q}}

\newcommand{\degree}{^\circ}

% Counting
\newcommand\perm[2][^n]{\prescript{#1\mkern-2.5mu}{}P_{#2}}
\newcommand\comb[2][^n]{\prescript{#1\mkern-0.5mu}{}C_{#2}}

\setlength\parindent{0pt}

% Card Symbols
\newcommand{\clubs}{\clubsuit}
\newcommand{\diamonds}{\vardiamondsuit}
\newcommand{\hearts}{\varheartsuit}
\newcommand{\spades}{\spadesuit}

%++++++++++++++++++++++++++++++++++++++++
\title{
    \vspace{2in}
    \textmd{\textbf{\courseNumber: \courseName}}
    \normalsize\vspace{0.1in}\\
    \vspace{0.1in}\Large{\text{\psetName}} \\
    \vspace{0.1in}\large{\text{\professor}}
    \vspace{3in}
}

\author{\name}
\date{\dueDate}

\begin{document}
    \maketitle
    \thispagestyle{empty}
    \pagebreak

    \section{Function Review}
    \begin{enumerate}[(1)]
        \item \textbf{Consider the relationships below. None of them are functions. For each example, explain why.}
        \begin{enumerate}[(a)]
            \item $f: \set*{a, b, c} \to \nats$ 
            \begin{figure}[H]
                \centering
                \begin{tabular}{c|c}
                    $x$ & $f(x)$ \\
                    \hline
                    a & 12 \\
                    b & 7 \\
                    b & 4 \\
                    c & 128
                \end{tabular}
            \end{figure}
            \begin{proof}
                The definition of a function is that all elements in the domain have a unique image in the codomain. In this case, $b$ has two images, $7$ and $4$, and is therefore $f$ is not a function.
            \end{proof}
            \item $g: \nats \to \nats , g(n) = n-4$
            \begin{proof}
                The definition of a function is that all elements in the domain have a unique image in the codomain. In this case, the preimages 1,2,3, and 4 map to numbers not in the codomain of $\nats$. Therefore, for these values, there is no unique image for all values in the domain, meaning $g$ is not a function.
            \end{proof}
            \item $h: \ints \to \rats, h(z) = \df{1}{z}$
            \begin{proof}
                The definition of a function is that all elements in the domain have a unique image in the codomain. In this case, the preimage of 0 maps to an undefined value, and therefore there is no unique image for all values in the domain, meaning $h$ is not a function.
            \end{proof}
        \end{enumerate}
        \item \textbf{Consider the function from $\nats$ to $\ints$ that multiplies every number in the source by 2. Is this function injective? Is it surjective?}
        \begin{proof}
            \textbf{The function is injective}. The definition of injective for a function $f$ with domain $A$ is \[ \forall x \forall y((x,y \in A \land f(x) = f(y)) \to x = y) \] In this case, $f(x) = 2x$ and $f(y) = 2y$, so if $f(x) = f(y)$, then $2x = 2y \to x = y$. Therefore, the function is injective.
        \end{proof}
        \begin{proof}
            \textbf{The function is not surjective}. The definition of surjective for a function $f$ with domain $A$ is \[ \forall y \exists x(x \in A \land f(x) = y) \] In this case, $f(x) = 2x$. This means that all values in the domain map to an even number. Since the codomain is $\ints$, there are infinitely many integers that are not even, there are infinitely many values in the codomain that are not images of the domain. Therefore, the function is not surjective.
        \end{proof}
        \item \textbf{Consider the function from $\nats$ to $\nats$ that adds one to every number in the source. Is this function injective? Surjective?}
        \begin{proof}
            \textbf{The function is injective}. The definition of injective for a function $f$ with domain $A$ is \[ \forall x \forall y((x,y \in A \land f(x) = f(y)) \to x = y) \] In this case, $f(x) = x+1$ and $f(y) = y+1$, so if $f(x) = f(y)$, then $x+1 = y+1 \to x = y$. Therefore, the function is injective.
        \end{proof}
        \begin{proof}
            \textbf{The function is not surjective}. The definition of surjective for a function $f$ with domain $A$ is \[ \forall y \exists x(x \in A \land f(x) = y) \] In this case, $f(x) = x+1$. To obtain an image of 1, the preimage must be 0. However, 0 is not in the domain of the function, $\nats$, meaning $\exists y \forall x (x \in A \land f(x) \neq y)$. Therefore, the function is not surjective.
        \end{proof}
        \item \textbf{Consider the function from $\ints$ to $\ints$ that adds one to every number in the source. Is this function injective? Surjective?}
        \begin{proof}
            \textbf{The function is injective}. The definition of injective for a function $f$ with domain $A$ is \[ \forall x \forall y((x,y \in A \land f(x) = f(y)) \to x = y) \] In this case, $f(x) = x+1$ and $f(y) = y+1$, so if $f(x) = f(y)$, then $x+1 = y+1 \to x = y$. Therefore, the function is injective.
        \end{proof}
        \begin{proof}
            \textbf{The function is surjective}. The definition of surjective for a function $f$ with domain $A$ is \[ \forall y \exists x(x \in A \land f(x) = y) \] In this case, $f(x) = x+1$. For all elements in the codomain, there exists a preimage in the domain, $x = y-1$. Therefore, the function is surjective.
        \end{proof}
        \item \textbf{Consider the function from $\ints$ to $\reals$ that sends ever integer to itself. Is this function injective? Surjective?}
        \begin{proof}
            \textbf{The function is injective}. The definition of injective for a function $f$ with domain $A$ is \[ \forall x \forall y((x,y \in A \land f(x) = f(y)) \to x = y) \] In this case, $x \in \ints \to f(x) = x$. Let $a,b \in \ints$. As $f(a) = a$ and $f(b) = b$, $f(a) = f(b) \to a = b$. Therefore, the function is injective.
        \end{proof}
        \begin{proof}
            \textbf{The function is not surjective}. The definition of surjective for a function $f$ with domain $A$ is \[ \forall y \exists x(x \in A \land f(x) = y) \] As the domain is $\ints$, the domain maps solely to integers in the codomain. However, the codomain is $\reals$, which contains infinitely many real numbers that are not integers. As $\exists y \forall x (x \in A \land f(x) \neq y)$, the function is not surjective.
        \end{proof}
        \item \textbf{Consider the function from $\nats$ to $\nats$ that subtracts one from every number other than 1, and sends 1 to 1. Is this function injective? Surjective?}
        \begin{proof}
            \textbf{The function is injective}. The definition of injective for a function $f$ with domain $A$ is \[ \forall x \forall y((x,y \in A \land f(x) = f(y)) \to x = y) \]
            We can represent the function as a piecewise function:
            \[ f: \nats \to \nats, f(x) = \begin{cases} 1 & x = 1 \\ x-1 & x \neq 1 \end{cases} \]
            The union of the domains of the two pieces is the domain of the function, $\nats$. The intersection of the domains of the two pieces is the empty set, so the function is injective.
        \end{proof}
        \begin{proof}
            \textbf{The function is surjective}. The definition of surjective for a function $f$ with domain $A$ is
            \[ \forall y \exists x (x \in A \land f(x) = y) \]
            When $x \neq 1$, $f(x) = x - 1$. For all values $y$ in the codomain, there is a value $x$ in the domain, $y+1$, that maps to it. When $x=1$, $y=1$. Despite the image 1 having 2 preimages, 1, and 2, the definition of surjective is still satisfied.
        \end{proof}
        \item \begin{enumerate}[(a)]
            \item \textbf{Write down a function with source $\nats$ and target $\{\spades, \hearts, \diamonds, \clubs\}$ which is not surjective.}
            \[ f: \nats \to \set*{\spades, \hearts, \diamonds, \clubs}, f(x) = \spades \]
            \item \textbf{Are there any injective functions from $\nats$ to $\{\spades, \hearts, \diamonds, \clubs\}$? Give an example or very briefly explain why not.}
            \begin{proof}
                \textbf{There are no injective functions from $\nats$ to $\set*{\spades, \hearts, \diamonds, \clubs}$}. We prove this by the pigeonhole principle. Let the values in the codomain be holes, and let the values in the domain be pigeons. There are 4 holes. With 4 pigeons, we can place each pigeon in each hole. With 5 pigeons, there must be a hole that contains at least 2---with 5 pigeons, the function is not injective. As the cardinality of $\nats > 5$, there will be more pigeons than holes, meaning at least 1 hole contains more than 1 pigeons. Thus, there is no possible injective function.
            \end{proof}
            \item \textbf{Are there any injective functions from $\{\spades, \hearts, \diamonds, \clubs\}$ to $\nats$ which are not surjective? Give an example or very briefly explain why not.}
            \end{enumerate}
    \end{enumerate}
\end{document}
