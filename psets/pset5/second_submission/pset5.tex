%++++++++++++++++++++++++++++++++++++++++
\documentclass[article, 12pt]{article}
\usepackage{float}
\usepackage{setspace}
\usepackage{tabu} % extra features for tabular environment
\usepackage{amsmath}  % improve math presentation
\usepackage{graphicx} % takes care of graphic including machinery
\usepackage[margin=1in]{geometry} % decreases margins
\usepackage{cite} % takes care of citations
\usepackage[final]{hyperref} % adds hyper links inside the generated pdf file
\usepackage{tikz}
\usepackage{caption} 
\usepackage{fancyhdr}
\usepackage{amssymb} % symbols like /therefore
\usepackage{amsthm} % proofs
\usepackage{enumerate} % lettered lists
\usepackage{mathtools} % macros
\usepackage[ all]{xy} % for diagrams
\usepackage{tkz-graph}

\usetikzlibrary{scopes}
% \usepackage{xcolor} \pagecolor[rgb]{0.12549019607,0.1294117647,0.13725490196} \color[rgb]{0.82352941176,0.76862745098,0.62745098039} % dark theme
\theoremstyle{definition}
\newtheorem{example}{Example}[subsubsection]
\newtheorem*{remark}{Remark}
\newtheorem{theorem}{Theorem}[subsubsection]
\newtheorem{definition}{Definition}[subsubsection]
\newtheorem{corollary}{Corollary}[subsubsection]
\hypersetup{
	colorlinks=false,      % false: boxed links; true: colored links
	linkcolor=blue,        % color of internal links
	citecolor=blue,        % color of links to bibliography
	filecolor=magenta,     % color of file links
	urlcolor=blue         
}
\usepackage{physics}
\usepackage{siunitx}
\usepackage{tikz,pgfplots}
\usepackage[outline]{contour} % glow around text
\usetikzlibrary{calc}
\usetikzlibrary{angles,quotes} % for pic
\usetikzlibrary{arrows.meta}
\tikzset{>=latex} % for LaTeX arrow head
\contourlength{1.2pt}

\colorlet{xcol}{blue!70!black}
\colorlet{vcol}{green!60!black}
\colorlet{myred}{red!70!black}
\colorlet{myblue}{blue!70!black}
\colorlet{mygreen}{green!70!black}
\colorlet{mydarkred}{myred!70!black}
\colorlet{mydarkblue}{myblue!60!black}
\colorlet{mydarkgreen}{mygreen!60!black}
\colorlet{acol}{red!50!blue!80!black!80}
\tikzstyle{CM}=[red!40!black,fill=red!80!black!80]
\tikzstyle{xline}=[xcol,thick,smooth]
\tikzstyle{mass}=[line width=0.6,red!30!black,fill=red!40!black!10,rounded corners=1,
                  top color=red!40!black!20,bottom color=red!40!black!10,shading angle=20]
\tikzstyle{faded mass}=[dashed,line width=0.1,red!30!black!40,fill=red!40!black!10,rounded corners=1,
                        top color=red!40!black!10,bottom color=red!40!black!10,shading angle=20]
\tikzstyle{rope}=[brown!70!black,very thick,line cap=round]
\def\rope#1{ \draw[black,line width=1.4] #1; \draw[rope,line width=1.1] #1; }
\tikzstyle{force}=[->,myred,very thick,line cap=round]
\tikzstyle{velocity}=[->,vcol,very thick,line cap=round]
\tikzstyle{Fproj}=[force,myred!40]
\tikzstyle{myarr}=[-{Latex[length=3,width=2]},thin]
\def\tick#1#2{\draw[thick] (#1)++(#2:0.12) --++ (#2-180:0.24)}
\DeclareMathOperator{\sn}{sn}
\DeclareMathOperator{\cn}{cn}
\DeclareMathOperator{\dn}{dn}
\def\N{80} % number of samples in plots


\usepackage{titling}
\renewcommand\maketitlehooka{\null\mbox{}\vfill}
\renewcommand\maketitlehookd{\vfill\null}
\usepackage{siunitx} % units
\usepackage{verbatim} 
\newcommand{\courseNumber}{MATH 1700}
\newcommand{\courseName}{Ideas in Mathematics}
\newcommand{\submission}{Second Submission}
\newcommand{\professor}{Professor Rimmer}
\newcommand{\psetName}{Worksheet 5: Uncountable Sets and Cantor's Diagonal Argument}
\newcommand{\dueDate}{Due: March 3, 2023}
\newcommand{\name}{Denny Cao}
\pagestyle{fancy}
\fancyhf{}% clears all header and footer fields
\fancyfoot[C]{--~\thepage~--}
\renewcommand*{\headrulewidth}{0.4pt}
\renewcommand*{\footrulewidth}{0pt}
\lhead{\name}
\chead{\courseNumber: \courseName}
\rhead{\professor}

% new theorem for questions and answers

\newtheorem{question}{Question}

\newtheorem{answer}{Answer}

\fancypagestyle{plain}{%
  \fancyhf{}% clears all header and footer fields
  \fancyfoot[C]{--~\thepage~--}%
  \renewcommand*{\headrulewidth}{0pt}%
  \renewcommand*{\footrulewidth}{0pt}%
}

% Shortcuts
\DeclarePairedDelimiter\ceil{\lceil}{\rceil} % ceil function
\DeclarePairedDelimiter\floor{\lfloor}{\rfloor} % floor function

\DeclarePairedDelimiter\paren{(}{)} % parenthesis

\newcommand{\df}{\displaystyle\frac} % displaystyle fraction
\newcommand{\qeq}{\overset{?}{=}} % questionable equality

\newcommand{\Mod}[1]{\;\mathrm{mod}\; #1} % modulo operator

\newcommand{\comp}{\circ} % composition

% Sets
\DeclarePairedDelimiter\set{\{}{\}}
\newcommand{\unite}{\cup}
\newcommand{\inter}{\cap}

\newcommand{\reals}{\mathbb{R}} % real numbers: textbook is Z^+ and 0
\newcommand{\ints}{\mathbb{Z}}
\newcommand{\nats}{\mathbb{N}}
\newcommand{\rats}{\mathbb{Q}}

\newcommand{\degree}{^\circ}

% Counting
\newcommand\perm[2][^n]{\prescript{#1\mkern-2.5mu}{}P_{#2}}
\newcommand\comb[2][^n]{\prescript{#1\mkern-0.5mu}{}C_{#2}}

% Relations
\newcommand{\rel}{\mathcal{R}} % relation

\setlength\parindent{0pt}

% Directed Graphs
\usetikzlibrary{arrows}
\tikzset{vertex/.style = {shape=circle,draw, minimum size=1.5em,
inner sep=0pt, outer sep=0pt}}
\tikzset{edge/.style = {->,> = latex'}}

% Contradiction
\newcommand{\contradiction}{{\hbox{%
    \setbox0=\hbox{$\mkern-3mu\times\mkern-3mu$}%
    \setbox1=\hbox to0pt{\hss$\times$\hss}%
    \copy0\raisebox{0.5\wd0}{\copy1}\raisebox{-0.5\wd0}{\box1}\box0
}}}

% Sign Charts
\newdimen\tcolw \tcolw=2.5em % the column width
\edef\ecatcode{\catcode`&=\the\catcode`&\relax}\catcode`&=4
\def\sgchart#1#2{\vbox{\offinterlineskip\halign{\hfil##\quad&##\hfil\crcr\sgchartA#2,:,%
   \omit\sgchartR&\kern.2pt\sgchartS{.5\tcolw}\relax\sgchartE#1,\relax,%
   \sgchartS{.5\tcolw}\relax\cr
   \noalign{\kern2pt}&\def~{}\kern.5\tcolw\sgchartD#1,\relax,\cr}}}
\def\sgchartA#1:#2,{\cr\ifx,#1,\else $#1$&\sgchartB#2{}\expandafter\sgchartA\fi}
\def\sgchartB#1{\hbox to\tcolw{\hss$#1$\hss}\sgchartC}
\def\sgchartC#1{\ifx,#1,\else
   \strut\vrule\kern-.4pt\hbox to\tcolw{\hss$#1$\hss}\expandafter\sgchartC\fi}
\def\sgchartD#1#2,{\ifx\relax#1\else\hbox to\tcolw{\hss$#1#2$\hss}\expandafter\sgchartD\fi}
\def\sgchartE#1#2,{\ifx\relax#1\else
    \ifx~#1\sgchartS\tcolw\circ \else\sgchartS\tcolw\bullet\fi \expandafter\sgchartE\fi}
\def\sgchartR{\leaders\vrule height2.8pt depth-2.4pt\hfil}
\def\sgchartS#1#2{\hbox to#1{\kern-.2pt\sgchartR \ifx\relax#2\else
   \kern-.7pt$#2$\kern-.7pt\sgchartR\fi\kern-.2pt}}
\ecatcode
%++++++++++++++++++++++++++++++++++++++++
% title stuff

\makeatletter
\renewcommand{\maketitle}{\bgroup\setlength{\parindent}{0pt}
    \begin{flushleft}
        \textbf{\@title} \\ \vskip0.2cm
        \begingroup
            \fontsize{14pt}{12pt}\selectfont
            \textbf{\submission}
            \vskip0.3cm
            \courseNumber: \courseName 
            \vskip0.3cm 
            \professor
        \endgroup \vskip0.3cm
        \dueDate \hfill\rlap{}\bf{\name} \\ \vskip0.1cm
        \hrulefill
    \end{flushleft}\egroup 
}
\makeatother

\title{\large\bf{\psetName}}

\begin{document}
    \maketitle
    \thispagestyle{empty}

    \section{Warm-Up: X's and O's}
    
    % Question 1
    \begin{question}
        Consider the following two-player game, which we will call Cantor's Game.
        \\[12pt]
        Player 1 begins by writing a sequence of $X$'s and $O$'s in the top row of the grid below.
        Player 2 then writes either an $X$ or an $O$ in the first box on the bottom. Player 1 then
        writes a sequence of $X$'s and $O$'s in the second row of the grid. Player 2 writes an $X$ or $O$ in the second bottom box. The players continue until all boxes are filled. Player 1 wins if the sequence on the bottom exactly matches any of the sequences Player 1 has written in
        the grid. Player 2 wins otherwise.
        \begin{enumerate}[a)]
            \item Which player has a winning strategy, and why?
            \item How does that strategy relate to Cantor's diagonal argument?
        \end{enumerate}
    \end{question}

    % Answer 1
    \begin{answer} \ 
        \begin{enumerate}[a)] 
            \item Player 2 has the winning strategy; they pick the symbol opposite the one placed by Player 1 each turn. 
            The sequence on the bottom will match one of the sequences that Player 1 has written, so Player 1 wins.
            \item The strategy in a) ensures that, the sequence in the bottom is different from the $i$th sequence in the $i$th column, thus ensuring that the sequence on the bottom is not equal to any of the sequences that Player 1 has written. This is the same strategy that Cantor used in his diagonal argument, proposing that the diagonal is a unique sequence that is not equal to any of the sequences in the grid.
        \end{enumerate}
    \end{answer}

    % Question 2
    \begin{question}
        Prove that the set of all infinite sequences of $X$'s and $O$'s is uncountable.
    \end{question}
    
    % Answer 2
    \begin{answer}
        
        \begin{proof} By contradiction. 
            \\[12pt]
            Assume for purposes of contradiction that the set of all infinite sequences of $X$'s and $O$'s is countable. Then, the subset of all sequences that begin with $X$ would also be countable (The subset of a countable set is countable). Under this assumption, the sequences that begin with $X$ can be listed in some order, $s_1, s_2, s_3, \dots$. We can write the sequences in the form $s_i = X, d_{i1}, d_{i2}, d_{i3}, d_{i4}, \dots$, where $d_{ij} \in \set*{X,O}$:
            \begin{align*}
                s_1 &= X,d_{11},d_{12},d_{13},d_{14}\dots \\
                s_2 &= X,d_{21},d_{22},d_{23},d_{24}\dots \\
                s_3 &= X,d_{31},d_{32},d_{33},d_{34}\dots \\
                s_4 &= X,d_{41},d_{42},d_{43},d_{44}\dots \\
                    &\vdotswithin{=}
            \end{align*}
            Then, form a new sequence $s = X, d_1, d_2, d_3, d_4, \dots$, where $d_i$ is determined by the following rule:
            \begin{equation*}
                d_i = \begin{cases}
                    X & \text{if } d_{ii} = O \\
                    O & \text{if } d_{ii} = X
                \end{cases}
            \end{equation*}
            For instance, if $s_1 = X, X, O, O, X, O, O, X, \dots, s_2 = X, O, O, X, O, O, X, \dots,$ \\ $s_3 = X, O, O, X, X, X, X, \dots, s_4 = X, X, X, X, O, O, X, \dots$, and so on. Then, we have $s = X, d_1, d_2, d_3, d_4, \dots = X, O, X, O, X, \dots$, where $d_1 = O$ because $d_{11} = X$, $d_2 = X$ because $d_{22} = O$, $d_3 = O$ because $d_{33} = X$, and so on. 
            \\[12pt]
            Therefore, $s$ is not equal to any of the sequences $s_1, s_2, s_3, \dots$ because $s$ differs from the $i$th sequence in the $i$th position. 
            \\[12pt]
            Because there exists a sequence $s$ that is not in the list, the assumption that all the sequences of $X$'s and $O$'s can be listed must be false. $\contradiction$ 
            \\[12pt]
            Thus, the set of all infinite sequences of $X$'s and $O$'s with $X$ as the first term cannot be listed, and is therefore uncountable. Any set with an uncountable subset is uncountable. Hence, the set of all infinite sequences of $X$'s and $O$'s is uncountable.
        \end{proof}
    \end{answer}

    % Question 3
    \begin{question}
        Prove that the set of real numbers between 0 and 0.0001 is uncountable.
    \end{question}

    % Answer 3
    \begin{answer}
        \begin{proof} By contradiction.
            \\[12pt]
            Assume for purposes of contradiction that the set of real numbers between 0 and 0.0001 is countable. Then, the set of all real numbers between 0 and 0.0001 can be listed in some order, $r_1, r_2, r_3, \dots$. We can write the decimal representations of the numbers in the form $r_i = 0.d_{i1}d_{i2}d_{i3}d_{i4}\dots$, where $d_{ij} \in \set*{k \mid 0 \leq k \leq 9, k \in \ints}$:
            \begin{align*}
                r_1 &= 0.d_{11}d_{12}d_{13}d_{14}\dots \\
                r_2 &= 0.d_{21}d_{22}d_{23}d_{24}\dots \\
                r_3 &= 0.d_{31}d_{32}d_{33}d_{34}\dots \\
                r_4 &= 0.d_{41}d_{42}d_{43}d_{44}\dots \\
                    &\vdotswithin{=}
            \end{align*}
            Then, form a new sequence $r = 0.d_1d_2d_3d_4\dots$, where $d_i$ is determined by the following rule:
            \begin{equation*}
                d_i = \begin{cases}
                    5 & \text{if } d_{ii} \neq 5 \\
                    4 & \text{if } d_{ii} = 5
                \end{cases}
            \end{equation*}
            Such a sequence $r$ is not equal to any of the sequences $r_1, r_2, r_3, \dots$ because $r$ differs from the $i$th sequence in the $i$th position.
            \\[12pt]
            Because there exists a sequence $r$ that is not in the list, the assumption that all the sequences of $d_{ij}$'s can be listed must be false. $\contradiction$
            \\[12pt]
            Thus, the set of all real numbers between 0 and 0.0001 cannot be listed, and is therefore uncountable. 
        \end{proof}
    \end{answer}

    \section{Countable and Uncountable Sets}
    
    % Question 4
    \begin{question}
        The following quote is from John Green's \textit{The Fault in Our Stars.} (Apologies for the spoiler).
        \begin{quotation}
            \noindent``There are infinite numbers between 0 and 1. There's .1 and .12 and .112 and an infinite collection of others. Of course, there is a bigger infinite set of numbers between 0 and 2, or between 0 and a million. Some infinities are bigger than other infinities. A writer we used to like taught us that. There are days, many of them, when I resent the size of my unbounded set. I want more numbers than I'm likely to get, and God, I want more numbers for Augustus Waters than he got. But, Gus, my love, I cannot tell you how thankful I am for our little infinity. I wouldn't trade it for the world. You gave me a forever within the numbered days, and I'm grateful.''
        \end{quotation}
        What is wrong with the narrator's understanding of cardinality?
    \end{question}

    % Answer 4
    \begin{answer}
        The narrator is incorrect when stating that the set of numbers between 0 and 2 is bigger than the set of numbers between 0 and 1. They have the same cardinality, as a bijection can be made by multiplying each number between 0 and 1 by 2. 
    \end{answer}

    % Question 5
    \begin{question}
        You have just graduated from Penn and begun your career at Very Large Corporation, Inc.
        Your new company is so large that its headquarters has (countably) infinitely many floors.
        On your first day of work, you climb up all the stairs, stopping on each floor to meet either 1, 2, or 3 of your new coworkers who work on that floor.

        \begin{enumerate}[a)]
            \item Do you greet countably many or uncountably many coworkers? Fully explain your
            answer.
            \item As you climb, you make a list of how many coworkers you have greeted on each floor. Is the set of possible lists countable or uncountable? Fully explain your answer.
        \end{enumerate}
    \end{question}

    % Answer 5
    \begin{answer} \
        \begin{enumerate}[a)]
            \item The number of coworkers I greet is countably infinite. On the first floor, there may be at most three coworkers to meet. On the second floor, there may be at most six coworkers to meet, since each of the one, two, or three coworkers on the second floor would have one, two, or three coworkers respectively to meet on the first floor. Continuing this process, on the $n$-th floor there may be at most $3n$ coworkers to meet. As $n$ is a natural number, a bijection can be made between the set of natural numbers and the set of all possible numbers of coworkers I greet by mapping $n$ to $3n$. Thus, the maximum number of coworkers I greet has the same cardinality as the set of natural numbers, which is countable.

            \item The set of possible lists of how many coworkers I greeted on each floor is uncountable. 
            \begin{proof} By contradiction.
                \\[12pt]
                Assume for purposes of contradiction that the set of possible lists of how many coworkers I greeted on each floor is countable. Then, the set of all possible lists of how many coworkers I greeted on each floor can be listed in some order, $l_1, l_2, l_3, \dots$. We can write the lists in the form $l_i = (n_{i1}, n_{i2}, n_{i3}, \dots)$, where $n_{ij} \in \set*{k \mid 0 \leq k \leq 3, k \in \ints}$:
                    \begin{align*}
                        l_1 &= (n_{11}, n_{12}, n_{13}, \dots) \\
                        l_2 &= (n_{21}, n_{22}, n_{23}, \dots) \\
                        l_3 &= (n_{31}, n_{32}, n_{33}, \dots) \\
                        l_4 &= (n_{41}, n_{42}, n_{43}, \dots) \\
                            &\vdotswithin{=}
                    \end{align*}
                Then, form a new list $l = (n_1, n_2, n_3, \dots)$, where $n_i$ is determined by the following rule:
                \begin{equation*}
                    n_i = \begin{cases}
                        2 & \text{if } n_{ii} \neq 2 \\
                        3 & \text{if } n_{ii} = 2
                    \end{cases}
                \end{equation*}
                Such a list $l$ is not equal to any of the lists $l_1, l_2, l_3, \dots$ because $l$ differs from the $i$th list in the $i$th position.
                \\[12pt]
                Because there exists a list $l$ that is not in the list, the assumption that all the lists of $n_{ij}$'s can be listed must be false. $\contradiction$
                \\[12pt]
                Thus, the set of all possible lists of how many coworkers I greeted on each floor cannot be listed, and is therefore uncountable.
            \end{proof}
        \end{enumerate}
    \end{answer}
    \section{Reflection}
    \textbf{Identify at least one wrong or failed idea that turned out to be helpful or enlightening in some way. For instance, that idea might have helped you solve a problem, or it may have been the start of a conversation that improved your understanding more generally. You can list one of your own ideas, or an idea that originated with a classmate. (Please give your classmate credit!)}

    After discussing with Larry Huang, I realized that, in Question 4, a bijection could actually be formed. Initially, I thought that both sets were uncountably infinite, and thus could not be compared. However, by multiplying each number in the set of numbers between 0 and 1 by 2, a bijection can be formed, meaning that the two sets have the same cardinality.
\end{document} 
